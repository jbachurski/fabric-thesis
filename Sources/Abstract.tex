\vspace*{1cm}
\begin{center}
  \Large
  \textbf{Breaking records: \\ Language design with structural subtyping}
\end{center}
\ifanonymised\else
\vspace{0.5cm}
\begin{center}
  \Large Jakub Bachurski
\end{center}
\fi
\vspace{1.5cm}
\begin{center}
  \Large
  \textsc{Abstract}
\end{center}
\vspace{0.2cm}
%
\begin{center}
\begin{minipage}{0.6\textwidth}
    \setlength{\parindent}{1em}
    Static type systems are the most widely deployed program verification technique. 
    And yet, much of today's programming happens in \emph{dynamically typed languages} which abandon static typing, trading away safety for flexibility. 
    This thesis argues for designing languages with \emph{structural subtyping} in mind, showing it mimics the flexibility of dynamically typed languages, while providing safety guarantees like traditional approaches. 

    To inform language design, reliable type inference is identified as key to programmer productivity. Based on Dolan's seminal invention of \emph{algebraic subtyping}, this thesis develops a constraint-based type inference framework supporting structural subtyping. Its expressive power is displayed in the design and implementation of a functional language with extensible data types. Lastly, structural subtyping is shown to resolve a long-standing problem in language design with the invention of a statically typed array programming calculus admitting type inference.

    This research aims to steer programming language design towards structural subtyping, and to lead to improvements in optional static type systems for dynamically typed languages, particularly type inference.
\end{minipage}
\end{center}