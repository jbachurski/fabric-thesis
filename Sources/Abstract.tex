\vspace*{2cm}
\begin{center}
  \Large
  \textbf{Breaking records: \\ Language design with structural subtyping}
\end{center}
\ifanonymised\else
\vspace{0.5cm}
\begin{center}
  \Large Jakub Bachurski
\end{center}
\fi
\vspace{1.5cm}
\begin{center}
  \Large
  \textsc{Abstract}
\end{center}
\vspace{0.2cm}
%
\begin{center}
\begin{minipage}{0.6\textwidth}
    \setlength{\parindent}{1em}
    Type systems are a staple of programming languages research. Their design balances the provided safety guarantees with the language's flexibility. This thesis argues for designing languages with \emph{structural subtyping} in mind. Structural subtyping provides the safety of traditional nominal typing, while mimicking the flexibility of widely-used dynamically typed languages. 

    To inform language design, reliable type inference is identified as key to programmer productivity. Based on Dolan's seminal \emph{algebraic subtyping} technique, this thesis develops a constraint-based type inference framework. Its expressive power is displayed in the design and implementation of a functional language with structural subtyping. Lastly, structural subtyping is shown to resolve long-standing problems in language design with the invention of a statically typed array programming calculus.

    This research aims to steer programming language design towards structural subtyping, and to lead to improvements in the inference of static types for dynamically typed programs.
\end{minipage}
\end{center}