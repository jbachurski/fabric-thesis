\vspace*{2cm}
\begin{center}
  \Large
  \textbf{Breaking records: \\ Language design with structural subtyping}
\end{center}
\ifanonymised\else
\vspace{0.5cm}
\begin{center}
  \Large Jakub Bachurski
\end{center}
\fi
\vspace{1.5cm}
\begin{center}
  \Large
  \textsc{Abstract}
\end{center}
\vspace{0.2cm}
%
\begin{center}
\begin{minipage}{0.55\textwidth}
    \setlength{\parindent}{1em}
    Static type systems are the most widely deployed program verification technique. 
    And yet, much of today's programming is with dynamically typed languages, which take little to no advantage of this technique. 
    This thesis argues for designing languages with \emph{structural subtyping} in mind. Structural subtyping provides the safety guarantees of traditional nominal approaches, while mimicking the flexibility of dynamically typed languages. 

    To inform language design, reliable type inference is identified as key to programmer productivity. Based on Dolan's seminal \emph{algebraic subtyping} technique, this thesis develops a constraint-based type inference framework that supports structural subtyping. Its expressive power is displayed in the design and implementation of a functional language with extensible data types. Lastly, structural subtyping is shown to resolve long-standing problems in language design with the invention of a statically typed array calculus.

    This research aims to steer programming language design towards structural subtyping, and to lead to improvements in optional static type systems and type inference for dynamically typed languages.
\end{minipage}
\end{center}