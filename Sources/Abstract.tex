\vspace*{1cm}
\begin{center}
  \Large
  \textbf{Breaking records: \\ Language design with structural subtyping}
\end{center}
\ifanonymised\else
\vspace{0.5cm}
\begin{center}
  \Large Jakub Bachurski
\end{center}
\fi
\vspace{1.5cm}
\begin{center}
  \Large
  \textsc{Abstract}
\end{center}
\vspace{0.2cm}
%
\begin{center}
\tolerance=1
\emergencystretch=\maxdimen
\hyphenpenalty=10000
\hbadness=10000
\begin{minipage}{0.61\textwidth}
    \setlength{\parindent}{1em}
    Static type systems are the most widely deployed program verification technique. 
    And yet, much of today's programming happens in \emph{dynamically typed languages}, trading away the safety of static typing for flexibility. 
    This thesis argues for designing languages with \emph{structural subtyping}, showing that it mimics the flexibility of dynamically typed languages, while providing safety guarantees like traditional type systems. 

    In order to inform language design, reliable type inference is identified as key to programmer productivity. 
    Based on Dolan's seminal invention of \emph{algebraic subtyping}, this thesis develops a constraint-based type inference framework supporting structural subtyping. 
    Its expressive power is displayed in the design and implementation~of a functional language with extensible data types. 
    Lastly, structural subtyping is shown to lead to a new middle ground between untyped and dependent type systems for array programming, with a novel design of a statically typed array calculus that admits type inference.

    This research aims to steer language design towards structural subtyping, and to lead to improvements in optional static type systems for dynamically typed languages.
\end{minipage}
\end{center}