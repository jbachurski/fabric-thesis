\chapter{Conclusions}
\label{conclusions}

I set out to show that structural subtyping is an interesting and rich approach to designing flexible type systems. I addressed this with techniques from both theory of programming languages and practical design.

From the theoretical side, I demonstrated that structural subtyping takes us closer towards the flexibility of dynamic languages by giving the FJ-FL translation (Chapter~\ref{static-soul}). 
I built a constraint-based type inference framework for languages with structural subtyping, \textbf{\inference{}}, deriving from \emph{algebraic subtyping} (Chapter~\ref{algebraic-subtyping}). 
It generalises prior work and has a simpler presentation, giving clear requirements for the type system.
By extending the type language of existing systems in a novel way (with \emph{homomorphism applications}), I unlocked a new solution to the record update problem. 
Thus, I widened the range of language features that can be conveniently type-checked without annotations.

From the practical side, I designed a functional language with structural subtyping -- \textbf{\fabric} (Chapter~\ref{fabric}). 
I implemented a compiler targeting \wasm{} for it -- \compiler{} -- including my type inference framework (\inference{}).
Lastly, I gave a novel array calculus design -- \textbf{\starr{}} (Chapter~\ref{star}). It uniquely features array shapes and indices that exploit structural subtyping, providing safety while admitting type inference with \inference{}. It thus serves as a new middle-ground between the conventional untyped discipline (flexible -- but unsafe), and complicated (but safe) dependently-typed systems. 

\paragraph{Publications} Besides the successful submission of the research paper on \starr{} to the 11th ARRAY Workshop, in April I gave a talk about the topic of my thesis at the British Colloquium for Theoretical Computer Science.

\paragraph{Further work}
I identify the following directions for further work as most interesting: \begin{itemize}
    \item Fully proving the correctness of constraint-based algebraic subtyping presentation -- particularly in presence of homomorphism applications. 
    \item Determining the scope of type system features expressible using homomorphisms, besides extensible records (FL) and type isomorphisms (\starr{}).
    \item With \inference{} developed, it would be exciting to use it to for typing real-world dynamic languages, like Python or Lua (as done with set-theoretic types \cite{set-theoretic-types-for-elixir, set-theoretic-types-for-erlang, castagna-dynamic}).
    \item Implementing \starr{} in practice (\eg{} extending Futhark \cite{futhark}, embedding in Python \cite{ein}), which leads to interesting problems: performant code generation, shape inference, and recursive indices.
\end{itemize}

Despite the age of this natural idea, I have shown structural subtyping should be revisited as an effective way for designing practical type systems. Likewise, it still leads to many interesting theoretical questions.