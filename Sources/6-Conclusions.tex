\chapter{Conclusions}
\label{conclusions}

I set out to show that structural subtyping is an interesting and rich approach to designing flexible type systems. For my approach, I applied both programming language theory and practical design.

From the theoretical side, I argued formally that structural subtyping takes us closer towards the flexibility of dynamic languages -- as opposed to nominally typed approaches -- by giving the FL-FJ translation (Chapter~\ref{static-soul}). I also contributed to the state-of-the-art of type inference under subtyping by formulating it as a general framework, \textbf{\inference{}} (Chapter~\ref{algebraic-subtyping}). 
I gave a novel extension to this approach -- hence unlocking a new solution to the record update problem. Thus, I widened the range of language features that can be type-checked without annotations -- mirroring the convenience of dynamically typed programming.

From the practical side, I proposed a design of a general-purpose language with structural subtyping -- \textbf{Fabric} (Chapter~\ref{fabric}). 
Furthermore, I gave a prototype implementation for its compiler, including my proposed approach to type inference, investigating a modern compilation target (\textsc{WebAssembly}).
Lastly, I gave a novel design for a statically typed array calculus, \textbf{Star}, which makes rich use of structural subtyping (Chapter~\ref{star}). Its design reveals an unexplored area in the space of array languages, serving as a middle-ground between an untyped (and thus unsafe) discipline, and complicated (but safe) dependently-typed systems. 

\paragraph{Further work} I identify the following directions for further work as most interesting: \begin{itemize}
    \item Formally proving the correctness of algebraic subtyping using the clear constraint-based presentation -- particularly for the extension of the type language with homomorphism applications. 
    \item Determining the scope of type system features expressible using homomorphisms: I only propose concrete uses for typing extensible records (FL) and expressing an isomorphism between types (Star).
    \item With the theory of type inference developed, it would be exciting to use it to for typing real-world languages -- particularly ones that provide optional type systems, like Python or Lua.
    \item Lastly, exploring Star further promises several interesting problems. Implementing and using it in practice (e.g.\@ extending Futhark, or by embedding in Python) might lead to interesting conclusions, and there still remain interesting theoretical problems to address -- like efficient code generation, shape inference and recursive index types.
\end{itemize}

Despite the age of this natural idea, I argue that structural subtyping should be revisited as an effective way for designing practical type systems. Likewise, it still leads to many interesting theoretical questions.