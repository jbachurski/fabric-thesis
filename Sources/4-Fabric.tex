\chapter{Design and Implementation of Fabric}
\label{fabric}

\begin{center}
    \Large \color{red} Chapter under construction
\end{center}

\todo[color=red]{}

\section{Design}

\cite{mlstruct, forsythe, pizza-java}
\todo[color=red]{}

\subsection{Data types}

\cite{dolan-thesis, mlstruct}
\todo[color=red]{}

\subsubsection{Rows}
\todo[color=red]{}

% \subsubsection{Patterns}
% \cite{parreaux-patterns}
% \todo[color=red]{}

\subsection{Nominal types} 

\cite{integrating-nominal-and-structural, structural-refinement-types}
\todo[color=red]{}

\section{Implementation}

For my implementation of \compiler{}, I used the OCaml language. Main dependencies included: \href{https://opensource.janestreet.com/core/}{Core}, \href{https://github.com/janestreet/ppx_jane}{\texttt{ppx\_jane}}, \href{https://github.com/mirage/alcotest}{\texttt{alcotest}} (test framework), and \href{https://github.com/inhabitedtype/angstrom}{\texttt{angstrom}} (parsing \fabric{} using parser combinators).

I summarise the contents of the repository in Table \ref{tab:repository},\todo[color=red]{}. The implementation is complete with only negligible omissions with respect to the specification. For brevity, I only describe some interesting aspects of type inference (Section \ref{subsec:type-inference-impl}) and code generation (Section \ref{subsec:codegen}).

\begin{table}[p]
    \centering
    \begin{tabular}{c|c}
         &  \\
         & 
    \end{tabular}
    \caption{The structure of my repository containing the implementation of \compiler{}.}
    \label{tab:repository}
\end{table}

\subsection{Type inference} 
\label{subsec:type-inference-impl}

I devised an implementation of \inference{} usable with any type system implementing its signature. 

\paragraph{Use of the module language} The framework was particularly satisfying to implement using OCaml's module language. Firstly, the framework's signature is given by an module \emph{signature} (given in Figure \ref{fig:inference-signature}). Type systems satisfying are specified by an \emph{implementation} module satisfying the signature, from which we can derive all aspects of constraint solving and simplification via a \emph{functor}.

\begin{figure}[p]
    \centering
    \begin{tabular}{c}
\begin{ocaml}
module type TypeSystem = sig
  type 'a typ  (* type constructors with holes 'a *)
  type arrow   (* homomorphisms *)

  (* mapping over type constructor subterms *)
  val map : f:('a -> 'b) -> 'a typ -> 'b typ
  (* type constructor lattice *)
  val top : 'a typ
  val bot : 'a typ
  val join : 'a lattice -> 'a typ -> 'a typ -> 'a typ
  val meet : 'a lattice -> 'a typ -> 'a typ -> 'a typ
  (* type constructor decomposition *)
  val decompose : ('a typ * 'a typ) -> ('a * 'a) list Or_error.t

  module Arrow : sig
    type t = arrow

    (* identity homomorphism *)
    val id : t
    val is_id : t -> bool
    (* composition *)
    val compose : t -> t -> t
    (* application to type constructors *)
    val apply : (t -> 'a -> 'a) -> t -> 'a typ -> 'a typ

    (* adjoint structure *)
    val swap_left : ('a typ -> 'a) -> t -> t * 'a
    val swap_right : ('a typ -> 'a) -> t -> t * 'a
  end
end
\end{ocaml}
\end{tabular}
    \caption{The key parts of the OCaml module signature of a type system suitable for type inference.}
    \label{fig:inference-signature}
\end{figure}

\paragraph{Type language representations} The OCaml version of the signature is polymorphic with respect to different type languages. This is because we have to change between different representations of types. 
Usually, we use the full type language (following \inference{}'s $\tau$), represented using the OCaml type \texttt{Alg.t}. On the other hand, in constraint solving (precisely as in Section \ref{sec:constraints}) we rely on the use of normal forms -- the clause-based CNF/DNF (represented by OCaml types \texttt{CNF.t}/\texttt{DNF.t}). As converting between the two can blow up the number of clauses, I try to keep subtyping constraints in the form $\tau_\mathrm{dnf} \sub \tau_\mathrm{cnf}$. To avoid code duplication, I implement DNF in terms of CNF, exploiting the duality between them. 

% \paragraph{Constraint solving} As advertised, the implementation of constraint solving is close to the technical description in Section \ref{sec:constraints}. It is infeasible to solve constraints directly by pattern matching on $\tau$/\texttt{Alg.t} due to mismatched-polarity cases like $\alpha \join \beta \sub \gamma \meet \delta$.

\paragraph{Simplification of type schemes} In Section \ref{subsec:simplification} I hinted at the possible approaches to simplifying the result of type inference. In \compiler{}, I implemented the following (among others): \begin{itemize}
    \item Simplification of CNF/DNF forms: removing subsumed clauses, removing  clauses equal to $\top$/$\bot$.
    \item Inlining bounds: if a type variable only occurs in covariant (or contravariant) positions, it can be replaced directly with the corresponding bound.
    \item Sandwich inequalities: if we find $\tau \sub \alpha \sub \tau$, then we can substitute $\alpha$ for $\tau$.
\end{itemize}

\paragraph{Constraint generation} To make implementing constraint generation $\cstr{\Gamma}{e}{\tau}$ more straightforward, I devised a simple DSL (example in Figure \ref{fig:cstr-gen}). I used OCaml's let-binding operators for composing constraints and introducing fresh type variables. 

\begin{figure}
    \centering
\begin{tabular}{c}
\begin{ocaml}
| Let (p, e, e') ->
    let* xs, t = pat p in
    let* e = go env e in
    let* () = e <: t in
    go (push xs env) e'
\end{ocaml}
\end{tabular}
    \caption{Example of constraint generation for \fabric{} let-bindings ($\fllet p e e'$). Here, \texttt{go env e} stands for a type $\tau$ such that $\texttt{env} \vdash \texttt{e} : \tau$ (under some carried constraint) -- a slight deviation from the specification.}
    \label{fig:cstr-gen}
\end{figure}

\subsection{Code generation} 
\label{subsec:codegen}

In order to execute \fabric{} code, I sought to generate code that could be lowered directly to machine code. I chose to go with \wasm{} \cite{wasm} and to use the \binaryen{} \cite{binaryen} toolchain, because it is modern, stable, relatively high-level, well-documented, and an open standard.\footnote{I also considered other targets -- C, Lua, .NET or JVM bytecode, or LLVM. Ultimately, the choice came down to what I thought would be most interesting and easiest to use.} Furthermore, I wanted to explore its new extension with garbage collection \cite{wasm-gc}, enabling automatic memory management for \fabric{}.

% I give a breakdown of the other targets I considered in Table \ref{tab:targets}.

% \begin{table}[]
%     \centering
%     \begin{tabular}{c|ccc}
%         Target & IR & Memory management & Stable & Toolchain \\ \hline
%         LLVM & SSA-based IR & No & No & C/C++ API
%         .NET & Stack & Yes & Yes & 
%         JVM & Stack & Yes & Yes &
%         C & Low-level language & No & Yes &
%         Lua & High-level language & Yes & Yes & 
%         \wasm{} & Stack/low-level & Yes & Yes & \binaryen{}, C/C++ API
%     \end{tabular}
%     \caption{Caption}
%     \label{tab:targets}
% \end{table}
% From the potential targets (of which I also considered LLVM, {.}NET bytecode, JVM bytecode, C, and Lua), 

\subsubsection{\binaryendsl{}}
\binaryen{} is only accessible from OCaml using the C~API.\footnote{There are ready solutions -- like \href{https://github.com/grain-lang/binaryen.ml}{\texttt{binaryen.ml}} -- but outdated (e.g.\@ no \textsc{WasmGC}) or broken with newer \binaryen{} versions.} I used the brilliant work of \textcite{ocaml-ctypes} on \href{https://github.com/yallop/ocaml-ctypes}{\texttt{ocaml-ctypes}} to produce my own, type-safe bindings.

Invoking the bindings directly often led to clumsy, imperative code. Instead, I devised a DSL on top of \binaryen{}, dubbed \binaryendsl{}. The most prevalent abstraction is \texttt{Cell} (Figure \ref{fig:cell-def}), which encapsulates the different types of storage available in \wasm{}, making it easier to write generic helpers for code generation.
I give a basic example of using \binaryendsl{} in Figure \ref{fig:binaryer-example}. 

I used \binaryendsl{} to implement code generation for \fabric{}, and included simple unit tests.


\begin{figure}[p]
    \centering
\begin{tabular}{c}
\begin{ocaml}
type loc = Cell0.loc =
| Local of { idx : T.Index.t }
| Global of { name : string; mut : bool; handle : T.Global.t }
| Table of { name : string; idx : T.Expression.t }
| Address of {
    addr : T.Expression.t;
    size : uint32;
    offset : uint32;
    align : uint32;
    mem : string;
  }
| Struct of {
    target : T.Expression.t;
    struct_type : T.HeapType.t;
    field_idx : T.Index.t;
  }
| Array of {
    target : T.Expression.t;
    array_type : T.HeapType.t;
    idx : T.Expression.t;
  }

type t = { typ : T.Type.t; loc : Cell0.loc }


(* val ( ! ) : t -> expr *)
(* val ( := ) : t -> expr -> expr *) 

\end{ocaml}
\end{tabular}
    \caption{Definition of a \texttt{Cell.t} -- abstracting over locals, globals, tables, memory addresses, structure fields, and array elements. I also give the signature of its two primitives: read (\texttt{!}) and write (\texttt{:=}). \texttt{T.<thing>.t} is the OCaml type for \binaryen{}'s representation of \texttt{<thing>}s.}
    \label{fig:cell-def}
\end{figure}

\begin{figure}[p]
    \centering
\begin{tabular}{c}
\begin{ocaml}
(* Set up Binaryen context *) 
let (module Ctx) = context () in
let open Ctx in
feature C.Features.reference_types;
feature C.Features.gc;
Memory.set ~initial:10 ~maximum:10 ();
(* Declare this function may print *)
Function.import "print_i32" "spectest" "print_i32" Type.int32 Type.none;
(* Define the main function *)
let main =
  Function.make ~params:Type.none ~result:Type.none (fun _ ->
    let open Cell in
    (* Type foobar_t of a structure with fields foo and bar *)
    let foobar_t =
      Struct.t Type.[ ("foo", field ~mut:true int32); ("bar", field int32) ]
    in
    (* Define a local reference and access it as a foobar_t *)
    let q = local Type.anyref in
    let q_foo = Struct.cell foobar_t !q "foo" in
    let q_bar = Struct.cell foobar_t !q "bar" in
    Control.block
      [
        q :=
          Struct.make foobar_t
            [ ("foo", Const.i32' 42); ("bar", Const.i32' (1337 - 42)) ];
        q_foo := Operator.I32.(!q_foo + !q_bar);
        Function.call "print_i32" [ !q_foo ] Type.none;
      ])
in
(* Mark the main function *)
Function.export "main" main;
Function.start main;
assert (validate ());
interpret ();    
\end{ocaml}
\end{tabular}
    \caption{Basic program defined using \binaryendsl{}. Under the hood, it calls \binaryen{}'s API, producing \wasm{} code (given in Appendix \ref{subsec:codegen}) which prints \texttt{1337} at runtime.}
    \label{fig:binaryer-example}
\end{figure}

\subsubsection{Runtime representation under subtyping}

It is well-known that structural typing makes the problem of efficiently representing values at runtime more difficult, as a value might be used as any supertype \cite{tapl}. 

This problem also arises for \fabric{}'s record and variant types: 
\begin{description}
    \item[Variants] I implemented the same approach as \textcite{polymorphic-variants} -- variant values are represented as a pair of a 32-bit hash of its tag and a reference to its payload.
    \item[Records] Due to time (and space) constraints, I did not experiment with record representations, and stuck to a na\"ive one (with linear-time projection). One interesting direction would be the work of \textcite{remy-extensible-records}, where a record is a hash table with a pre-computed hashing function (so projection is constant-time). 
\end{description}

\section{Conclusions}

I summarise my experience with the three main areas covered in this chapter: the use and implementation of \inference{}, targeting \wasm{}, and the design of languages for algebraic subtyping.

\paragraph{Experience with \inference{}}
\textcite{pottier-framework} outlined the benefits of a general constraint-based type inference framework as a good way for researching language extensions. 
% His work resulted in \textsc{Inferno}, a framework for constraint-based Hindley-Milner type inference -- I performed a similar task in a setting with subtyping. 
By separating concerns of the type language and the constraint language in my implementation of \compiler{}, I was able to quickly experiment with type inference for not only \fabric{}, but also \starr{} (Chapter \ref{star}).

\paragraph{Experience with \wasm{}}
By getting hands-on experience with \wasm{}, I found that while its tooling for high-level languages is limited, bespoke solutions are possible -- e.g.\@ building on \textsc{Binaryen}. Its formal specification and stability are also great boons towards its practical use. However, existing tooling could improve its error diagnostics for invalid emitted programs, as debugging was difficult.

The current specification of reference types to only structures and arrays is limiting in terms of efficiency. Reference types are also opaque (e.g.\@ we cannot coerce them to integers like C-style pointers), making some efficient memory representations impossible \cite{double-ended-bit-stealing}.

\paragraph{Experience with algebraic subtyping} My biggest grudge with type inference is with the user-visible output of type inference -- type signatures -- as my implementation lacked in terms of simplification, confirming historical sentiment about inference with subtyping. I did not find the requirement of constructing a lattice of type constructors constraining during design, and it prevented me from constructing misbehaved type system features. Rapidly experimenting with new features using \inference{} was liberating.
