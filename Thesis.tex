\documentclass{mimosis}

\usepackage{metalogo}
\usepackage{tcolorbox}

\usepackage{tikz}

\usepackage{todonotes}

\usepackage{needspace}
\usepackage{etoolbox}

%%%%%%%%%%%%%%%%%%%%%%%%%%%%%%%%%%%%%%%%%%%%%%%%%%%%%%%%%%%%%%%%%%%%%%%%
% Listings
%%%%%%%%%%%%%%%%%%%%%%%%%%%%%%%%%%%%%%%%%%%%%%%%%%%%%%%%%%%%%%%%%%%%%%%%

\usepackage{listings}

\definecolor{deepblue}{rgb}{0,0,0.6}
\definecolor{deepred}{rgb}{0.6,0,0}
\definecolor{deepgreen}{rgb}{0,0.6,0}

\def\backtick{\`{}}
\newcommand\basestyle{\lstset{
    basicstyle=\small\ttfamily,
    morekeywords={self},
    keywordstyle=\color{deepblue},
    stringstyle=\color{deepgreen},
    showstringspaces=false,
    literate={`}{\backtick}1,
    escapechar=@
}}

\newcommand\ocamlstyle{\basestyle{} \lstset{language=caml}}
\lstnewenvironment{ocaml}[1][]{\ocamlstyle \lstset{#1}}{}

\newcommand\pythonstyle{\basestyle{} \lstset{language=Python}}
\lstnewenvironment{python}[1][]{\pythonstyle \lstset{#1}}{}

\newcommand\javastyle{\basestyle{} \lstset{language=Java}}
\lstnewenvironment{java}[1][]{\javastyle \lstset{#1}}{}

%%%%%%%%%%%%%%%%%%%%%%%%%%%%%%%%%%%%%%%%%%%%%%%%%%%%%%%%%%%%%%%%%%%%%%%%
% Nomenclature
%%%%%%%%%%%%%%%%%%%%%%%%%%%%%%%%%%%%%%%%%%%%%%%%%%%%%%%%%%%%%%%%%%%%%%%%
\usepackage[intoc, english]{nomencl}
\makenomenclature

% https://tex.stackexchange.com/questions/383027/nomencl-package-sort-by-order-of-appearance
\providetoggle{nomsort}
\settoggle{nomsort}{false}

\makeatletter
\iftoggle{nomsort}{%
    \let\old@@@nomenclature=\@@@nomenclature        
        \newcounter{@nomcount} \setcounter{@nomcount}{0}%
        \renewcommand\the@nomcount{\two@digits{\value{@nomcount}}}% 
        \def\@@@nomenclature[#1]#2#3{% 
          \addtocounter{@nomcount}{1}%
        \def\@tempa{#2}\def\@tempb{#3}%
          \protected@write\@nomenclaturefile{}%
          {\string\nomenclatureentry{\the@nomcount\nom@verb\@tempa @[{\nom@verb\@tempa}]%
          \begingroup\nom@verb\@tempb\protect\nomeqref{\theequation}%
          |nompageref}{\thepage}}%
          \endgroup
          \@esphack}%
      }{}
\makeatother

%%%%%%%%%%%%%%%%%%%%%%%%%%%%%%%%%%%%%%%%%%%%%%%%%%%%%%%%%%%%%%%%%%%%%%%%
% Hyperlinks & bookmarks
%%%%%%%%%%%%%%%%%%%%%%%%%%%%%%%%%%%%%%%%%%%%%%%%%%%%%%%%%%%%%%%%%%%%%%%%

\usepackage[%
  colorlinks = true,
  citecolor  = RoyalBlue,
  linkcolor  = RoyalBlue,
  urlcolor   = RoyalBlue,
  unicode,
  ]{hyperref}

\usepackage{bookmark}

%%%%%%%%%%%%%%%%%%%%%%%%%%%%%%%%%%%%%%%%%%%%%%%%%%%%%%%%%%%%%%%%%%%%%%%%
% Bibliography
%%%%%%%%%%%%%%%%%%%%%%%%%%%%%%%%%%%%%%%%%%%%%%%%%%%%%%%%%%%%%%%%%%%%%%%%
%
% I like the bibliography to be extremely plain, showing only a numeric
% identifier and citing everything in simple brackets. The first names,
% if present, will be initialized. DOIs and URLs will be preserved.

\usepackage[%
  autocite     = plain,
  backend      = biber,
  doi          = true,
  url          = true,
  giveninits   = true,
  hyperref     = true,
  maxbibnames  = 99,
  maxcitenames = 99,
  sortcites    = true,
  style        = numeric,
  ]{biblatex}

%%%%%%%%%%%%%%%%%%%%%%%%%%%%%%%%%%%%%%%%%%%%%%%%%%%%%%%%%%%%%%%%%%%%%%%%
% Some adjustments to make the bibliography more clean
%%%%%%%%%%%%%%%%%%%%%%%%%%%%%%%%%%%%%%%%%%%%%%%%%%%%%%%%%%%%%%%%%%%%%%%%
%
% The subsequent commands do the following:
%  - Removing the month field from the bibliography
%  - Fixing the Oxford commma
%  - Suppress the "in" for journal articles
%  - Remove the parentheses of the year in an article
%  - Delimit volume and issue of an article by a colon ":" instead of
%    a dot ""
%  - Use commas to separate the location of publishers from their name
%  - Remove the abbreviation for technical reports
%  - Display the label of bibliographic entries without brackets in the
%    bibliography
%  - Ensure that DOIs are followed by a non-breakable space
%  - Use hair spaces between initials of authors
%  - Make the font size of citations smaller
%  - Fixing ordinal numbers (1st, 2nd, 3rd, and so) on by using
%    superscripts

% Remove the month field from the bibliography. It does not serve a good
% purpose, I guess. And often, it cannot be used because the journals
% have some crazy issue policies.
\AtEveryBibitem{\clearfield{month}}
\AtEveryCitekey{\clearfield{month}}

% Fixing the Oxford comma. Not sure whether this is the proper solution.
% More information is available under [1] and [2].
%
% [1] http://tex.stackexchange.com/questions/97712/biblatex-apa-style-is-missing-a-comma-in-the-references-why
% [2] http://tex.stackexchange.com/questions/44048/use-et-al-in-biblatex-custom-style
%
\AtBeginBibliography{%
  \renewcommand*{\finalnamedelim}{%
    \ifthenelse{\value{listcount} > 2}{%
      \addcomma
      \addspace
      \bibstring{and}%
    }{%
      \addspace
      \bibstring{and}%
    }
  }
}

% Suppress "in" for journal articles. This is unnecessary in my opinion
% because the journal title is typeset in italics anyway.
\renewbibmacro{in:}{%
  \ifentrytype{article}
  {%
  }%
  % else
  {%
    \printtext{\bibstring{in}\intitlepunct}%
  }%
}

% Remove the parentheses for the year in an article. This removes a lot
% of undesired parentheses in the bibliography, thereby improving the
% readability. Moreover, it makes the look of the bibliography more
% consistent.
\renewbibmacro*{issue+date}{%
  \setunit{\addcomma\space}
    \iffieldundef{issue}
      {\usebibmacro{date}}
      {\printfield{issue}%
       \setunit*{\addspace}%
       \usebibmacro{date}}%
  \newunit}

% Delimit the volume and the number of an article by a colon instead of
% by a dot, which I consider to be more readable.
\renewbibmacro*{volume+number+eid}{%
  \printfield{volume}%
  \setunit*{\addcolon}%
  \printfield{number}%
  \setunit{\addcomma\space}%
  \printfield{eid}%
}

% Do not use a colon for the publisher location. Instead, connect
% publisher, location, and date via commas.
\renewbibmacro*{publisher+location+date}{%
  \printlist{publisher}%
  \setunit*{\addcomma\space}%
  \printlist{location}%
  \setunit*{\addcomma\space}%
  \usebibmacro{date}%
  \newunit%
}

% Ditto for other entry types.
\renewbibmacro*{organization+location+date}{%
  \printlist{location}%
  \setunit*{\addcomma\space}%
  \printlist{organization}%
  \setunit*{\addcomma\space}%
  \usebibmacro{date}%
  \newunit%
}

% Display the label of a bibliographic entry in bare style, without any
% brackets. I like this more than the default.
%
% Note that this is *really* the proper and official way of doing this.
\DeclareFieldFormat{labelnumberwidth}{#1\adddot}

% Ensure that DOIs are followed by a non-breakable space.
\DeclareFieldFormat{doi}{%
  \mkbibacro{DOI}\addcolon\addnbspace
    \ifhyperref
      {\href{http://dx.doi.org/#1}{\nolinkurl{#1}}}
      %
      {\nolinkurl{#1}}
}

% Use proper hair spaces between initials as suggested by Bringhurst and
% others.
\renewcommand*\bibinitdelim {\addnbthinspace}
\renewcommand*\bibnamedelima{\addnbthinspace}
\renewcommand*\bibnamedelimb{\addnbthinspace}
\renewcommand*\bibnamedelimi{\addnbthinspace}

% Make the font size of citations smaller. Depending on your selected
% font, you might not need this.
\usepackage{relsize}
\renewcommand*{\citesetup}{%
  \biburlsetup
  \relsize{-.5}%
}

\DeclareLanguageMapping{english}{english-mimosis}

% Make hyperlinks extend to the author name if `\textcite` is being used
% instead of another cite command.

\DeclareFieldFormat{citehyperref}{%
  % Need this to avoid nested links
  \DeclareFieldAlias{bibhyperref}{noformat}%
  \bibhyperref{#1}%
}

\DeclareFieldFormat{textcitehyperref}{%
  % Need this to avoid nested links
  \DeclareFieldAlias{bibhyperref}{noformat}%
  \bibhyperref{%
    #1%
    \ifbool{cbx:parens}
      {\bibcloseparen\global\boolfalse{cbx:parens}}
      {}%
    }%
}

\savebibmacro{cite}
\savebibmacro{textcite}

\renewbibmacro*{cite}{%
  \printtext[citehyperref]{%
    \restorebibmacro{cite}%
    \usebibmacro{cite}}%
}

\renewbibmacro*{textcite}{%
  \ifboolexpr{
    ( not test {\iffieldundef{prenote}} and
      test {\ifnumequal{\value{citecount}}{1}} )
    or
    ( not test {\iffieldundef{postnote}} and
      test {\ifnumequal{\value{citecount}}{\value{citetotal}}} )
  }%
  {\DeclareFieldAlias{textcitehyperref}{noformat}}
  {}%
  \printtext[textcitehyperref]{%
    \restorebibmacro{textcite}%
    \usebibmacro{textcite}}%
}
\addbibresource{Thesis.bib}

%%%%%%%%%%%%%%%%%%%%%%%%%%%%%%%%%%%%%%%%%%%%%%%%%%%%%%%%%%%%%%%%%%%%%%%%
% Fonts
%%%%%%%%%%%%%%%%%%%%%%%%%%%%%%%%%%%%%%%%%%%%%%%%%%%%%%%%%%%%%%%%%%%%%%%%

\ifxetexorluatex
  \usepackage{unicode-math}
  \setmainfont{EB Garamond}
  \setmathfont{Garamond Math}

  % Load some missing symbols from another font.
  \setmathfont{STIX Two Math}[%
    range = {
      \sharp,
      \natural,
      \flat,
      \clubsuit,
      \spadesuit,
      \checkmark
    }
  ]
  \setmonofont[Scale=MatchLowercase]{Source Code Pro}
\else
  \usepackage[lf]{ebgaramond}
  \usepackage[oldstyle,scale=0.9]{sourcecodepro}
  \singlespacing
\fi

% \newacronym{FL}{FL}{Featherweight Lua}
% \newacronym{FJ}{FJ}{Featherweight Java}

\makeindex
\makeglossaries

%%%%%%%%%%%%%%%%%%%%%%%%%%%%%%%%%%%%%%%%%%%%%%%%%%%%%%%%%%%%%%%%%%%%%%%%
% Ordinals
%%%%%%%%%%%%%%%%%%%%%%%%%%%%%%%%%%%%%%%%%%%%%%%%%%%%%%%%%%%%%%%%%%%%%%%%

\makeatletter
\@ifundefined{st}{%
  \newcommand{\st}{\textsuperscript{\textup{st}}\xspace}
}{}
\@ifundefined{rd}{%
  \newcommand{\rd}{\textsuperscript{\textup{rd}}\xspace}
}{}
\@ifundefined{nd}{%
  \newcommand{\nd}{\textsuperscript{\textup{nd}}\xspace}
}{}
\makeatother

\renewcommand{\th}{\textsuperscript{\textup{th}}\xspace}

%%%%%%%%%%%%%%%%%%%%%%%%%%%%%%%%%%%%%%%%%%%%%%%%%%%%%%%%%%%%%%%%%%%%%%%%
% Theorems
%%%%%%%%%%%%%%%%%%%%%%%%%%%%%%%%%%%%%%%%%%%%%%%%%%%%%%%%%%%%%%%%%%%%%%%%

\newtheorem{theorem}{Theorem}[chapter]

%%%%%%%%%%%%%%%%%%%%%%%%%%%%%%%%%%%%%%%%%%%%%%%%%%%%%%%%%%%%%%%%%%%%%%%%
% Incipit
%%%%%%%%%%%%%%%%%%%%%%%%%%%%%%%%%%%%%%%%%%%%%%%%%%%%%%%%%%%%%%%%%%%%%%%%

\title{Breaking records}
\subtitle{Language design with structural subtyping}
\author{Jakub Krzysztof Bachurski}

\newcommand{\fabric}{\textsc{Fabric}}
\newcommand{\inference}{\textsc{Comb}}
\newcommand{\compiler}{\textsc{Weaver}}

\begin{document}
%TC:ignore 
  \renewcommand\nomgroup[1]{%
  \item[\bfseries
  \ifstrequal{#1}{A}{Common symbols}{%
  \ifstrequal{#1}{B}{Common syntax}{%
  \ifstrequal{#1}{J}{Featherweight Java}{%
  \ifstrequal{#1}{L}{Featherweight Lua}{%
  \ifstrequal{#1}{P}{Type inference}{%
  \ifstrequal{#1}{S}{Subtyping and the type lattice}{%
  \ifstrequal{#1}{T}{Constraint-based algebraic subtyping}{%
  \ifstrequal{#1}{Y}{Fabric}{%
  \ifstrequal{#1}{Z}{Star}{%
  }}}}}}}}}%
]}

\makeatletter
\newcommand*\fsize{\dimexpr\f@size pt\relax}
\makeatother

\newcommand{\newnotation}[4]{\nomenclature[#4]{#2}{#3}}

\newcommand{\graintro}[1]{{#1} \,::=&\; }
\newcommand{\graitem}{\\ \mid&\;}
\newcommand{\irule}[3]{{\renewcommand\arraystretch{1.5} \begin{array}{ll} \textsc{#1} \\ \dfrac{\phantom| #2 \phantom|}{\phantom| #3 \phantom|}\end{array}}}
\newcommand{\defeq}{\stackrel{\Delta}{=}}

\newcommand{\ctx}[2]{{#1}\left\langle{#2}\right\rangle}
\newcommand{\ctxhole}{\diamond}

%%%%%%%%%%%%%%%%%%%%%%%%%%%%%%%%%%%%%%%%%%%%%%%%%%%%%%%%%%%%%%%%%%%%%%%%
% Symbols
%%%%%%%%%%%%%%%%%%%%%%%%%%%%%%%%%%%%%%%%%%%%%%%%%%%%%%%%%%%%%%%%%%%%%%%%

\newnotation{var}{$x, y, z$}{Variable}{A010}
\newnotation{tvar}{$\alpha, \beta, \gamma$}{Type variable}{A011}
\newnotation{expr}{$e$}{Expression}{A020}
\newnotation{type}{$\tau, \pi$}{Type}{A040}
\newnotation{label}{$\ell$}{Record label}{A045}
\newnotation{tag}{$T$}{Variant tag}{A046}
\newnotation{env}{$\Gamma$}{Typing environment}{A050}

\newcommand{\substx}[1]{\left[ #1 \right]}
\newcommand{\subst}[2]{\left[ #1/#2 \right]}
\newnotation{subst}{$\subst{e}{x}e$}{Substitution}{A060}
\newcommand{\tsubst}[3]{\subst{#1}{#2}{#3}}
\newnotation{tsubst}{$\subst{\tau}{\alpha}{\tau}$}{Type substitution, with $\psi ::= \cdot \mid \psi, \tau/\alpha$}{A061}

%%%%%%%%%%%%%%%%%%%%%%%%%%%%%%%%%%%%%%%%%%%%%%%%%%%%%%%%%%%%%%%%%%%%%%%%
% Common syntax
%%%%%%%%%%%%%%%%%%%%%%%%%%%%%%%%%%%%%%%%%%%%%%%%%%%%%%%%%%%%%%%%%%%%%%%%

\newnotation{def}{$\defeq$}{Definition}{B010}
\newnotation{rep}{$\overline t$}{Repetition of $t$}{B020}

\newcommand{\letbind}[2]{\mathrm{let}\,{#1}={#2}\,\mathrm{in}\,}
\newcommand{\letbindx}[2]{\mathrm{let}\,{#1}={#2}}
\newcommand{\fllet}[2]{\letbind{#1}{#2}}
\newcommand{\flletx}[2]{\letbindx{#1}{#2}}
% \newnotation{fllet}{$\mathrm{let}$}{Let-binding}{B030}

\newnotation{top}{$\top$}{Top type -- supertype of all types}{B048}
\newnotation{bot}{$\bot$}{Bottom type -- subtype of all types}{B049}

\newcommand{\lam}[1]{\lambda {#1} \ldotp}
\newcommand{\fllam}[1]{\lam{#1}}
% \newnotation{fllam}{$\lambda$}{Function ($\lambda$ abstraction)}{B040}

\newcommand{\rcd}[1]{\left\{ {#1} \right\}}
\newnotation{flrec}{$\rcd{\cdots}$}{Record}{B050}

\newcommand{\vrt}[1]{\left[ {#1} \right]}
\newnotation{flvrt}{$\vrt{\cdots}$}{Variant}{B050}

\newcommand{\denot}[1]{\left\llbracket{#1}\right\rrbracket}
\newcommand{\denotl}[1]{\left\lceil\!\!\left\lceil{#1}\right\rceil\!\!\right\rceil}
\newcommand{\denoty}[1]{\left(\!\left| #1
\right|\!\right)}
\newnotation{denot}{$\denot -$}{Expression translation: semantics- and type-preserving}{B064}
\newnotation{denotl}{$\denotl -$}{Expression translation: semantics-preserving}{B066}
\newnotation{denoty}{$\denoty -$}{Type translation}{B068}
\newnotation{typing}{$\Gamma \vdash e : \tau$}{Typing judgement}{B070}

%%%%%%%%%%%%%%%%%%%%%%%%%%%%%%%%%%%%%%%%%%%%%%%%%%%%%%%%%%%%%%%%%%%%%%%%
% Featherweight Lua
%%%%%%%%%%%%%%%%%%%%%%%%%%%%%%%%%%%%%%%%%%%%%%%%%%%%%%%%%%%%%%%%%%%%%%%%

\newnotation{flfieldtype}{$\phi$}{Field type}{L030}

\newcommand{\flrec}[1]{\rcd{#1}}

\newcommand{\flext}[2]{\left\{ {#1} \,\middle|\, {#2} \right\}}
\newnotation{flext}{$\flext{\overline{\ell = e}}{e}$}{Record extension}{L040}

\newnotation{flrectyp}{$\flext{\overline{\ell : \phi}}{\phi}$}{Record type}{L045}

\newcommand{\flftop}{\top \!\!\!\! \top}
\newcommand{\flfbot}{\bot \!\!\!\! \bot}
\newcommand{\flpresent}[1]{\boxed{#1}}
\newcommand{\flabsent}{\Box}
\newnotation{flftop}{$\flftop$}{Top field type}{L046}
\newnotation{flfbot}{$\flfbot$}{Bottom field type}{L047}
\newnotation{flpresent}{$\flpresent \tau$}{Present field type}{L048}
\newnotation{flabsent}{$\flabsent$}{Absent field type}{L049}

\newcommand{\flproj}[2]{{#1}{.}{#2}}
\newnotation{flproj}{$\flproj e \ell$}{Let-binding}{L050}

\newcommand{\flcast}[2]{{#1} \vartriangleright {#2}}
\newnotation{flcast}{$\flcast{e}{\tau}$}{Runtime type coercion}{L060}

\newcommand{\centredalign}{\hspace{.45\textwidth}\phantom{{}={}}&\phantom{{}={}}\hspace{.45\textwidth}}

\newcommand{\lua}[1]{\textsf{#1}}
\newnotation{lua}{$\lua e$}{Lua expression \emph{(written in sans-serif font)}}{L070}


%%%%%%%%%%%%%%%%%%%%%%%%%%%%%%%%%%%%%%%%%%%%%%%%%%%%%%%%%%%%%%%%%%%%%%%%
% Featherweight Java
%%%%%%%%%%%%%%%%%%%%%%%%%%%%%%%%%%%%%%%%%%%%%%%%%%%%%%%%%%%%%%%%%%%%%%%%

\newcommand{\fj}[1]{\texttt{#1}}
\newcommand{\fjoverline}[1]{$\overline{\mbox{#1}}$}

\newcommand{\fjconsdef}{\fj{C(\fjoverline{C f}) \{ super($\overline{\fj{f}'}$); \fjoverline{this.f=f;} \}}}
\newcommand{\fjmethdefx}{\fj{C}' \fj{ m(\fjoverline{C x})\{ ... \}}}
\newcommand{\fjmethdef}{\fj{C}' \fj{ m(\fjoverline{C x})\{ return e; \}}}
\newcommand{\fjclassdef}{\fj{class C extends C' \{ \fjoverline{C f}; K; \fjoverline{M}; \}}}

\newnotation{fj}{$\fj e$}{Expression \emph{(written in monospace font)}}{J010}
\newnotation{fjtype}{$\fj C$}{Class}{J011}
\newnotation{fjtype}{$\fj m$}{Method name}{J012}
\newnotation{fjtype}{$\fj f$}{Field, variable name}{J013}
\newnotation{fjconsdef}{$K$}{Class constructor definition}{J020}
\newnotation{fjmethdef}{$M$}{Method definition}{J030}
\newnotation{fjclassdef}{$L$}{Class definition}{J040}

%%%%%%%%%%%%%%%%%%%%%%%%%%%%%%%%%%%%%%%%%%%%%%%%%%%%%%%%%%%%%%%%%%%%%%%%
% Type inference
%%%%%%%%%%%%%%%%%%%%%%%%%%%%%%%%%%%%%%%%%%%%%%%%%%%%%%%%%%%%%%%%%%%%%%%%

% \newcommand{\asubst}{\psi}
% \newnotation{asubst}{$\asubst$}{Substitution, e.g.\@ in $\substx{\asubst} e = e$}{P010}

\newcommand{\sch}{\sigma}
\newnotation{sch}{$\sch$}{Type scheme}{P010}
\newnotation{inst}{$\sigma \models \tau$}{Type scheme instantiation}{P011}

\newcommand{\rec}[1]{\mu\,{#1}\ldotp}
\newnotation{rec}{$\rec \alpha \tau$}{Recursive type, where variable $\alpha$ refers to entire type when occuring in $\tau$}{P020}

%%%%%%%%%%%%%%%%%%%%%%%%%%%%%%%%%%%%%%%%%%%%%%%%%%%%%%%%%%%%%%%%%%%%%%%%
% Subtyping lattice
%%%%%%%%%%%%%%%%%%%%%%%%%%%%%%%%%%%%%%%%%%%%%%%%%%%%%%%%%%%%%%%%%%%%%%%%

\newcommand{\typeq}[0]{\equiv}
\newnotation{typeq}{$\typeq$}{Type equivalence under the Boolean algebra}{S000}

\newcommand{\sub}[0]{\leqslant}
\newnotation{sub}{$\sub$}{Subtype}{S010}

\newcommand{\super}[0]{\geqslant}
\newnotation{super}{$\super$}{Supertype}{S015}

\newcommand{\constr}[0]{K}
\newnotation{constr}{$\constr[\overline \tau]$}{Type constructor $\constr$ with type subterms $\overline \tau$}{S030}

\newcommand{\cmeet}[0]{\sqcap}
\newnotation{cmeet}{$\cmeet$}{Meet in the lattice of type constructors}{S031}

\newcommand{\cjoin}[0]{\sqcup}
\newnotation{cjoin}{$\cjoin$}{Join in the lattice of types constructors}{S032}

\newcommand{\cdecomp}[0]{\lhd}
\newnotation{cdecomp}{$\cdecomp$}{Decomposition of type constructor subtyping constraint}{S031}

\newcommand{\meet}[0]{\land}
\newnotation{meet}{$\meet$}{Meet in the lattice of types}{S035}

\newcommand{\join}[0]{\lor}
\newnotation{join}{$\lor$}{Join in the lattice of types}{S036}

\newcommand{\comp}[0]{\lnot}
\newnotation{comp}{$\comp$}{Complement in the Boolean algebra of types}{S040}

\newcommand{\subsume}[0]{\sub^\forall}
\newnotation{subsume}{$\subsume$}{Subsumption of type schemes}{S050}

\newcommand{\morph}{\theta}
\newnotation{morphism}{$\morph$}{Homomorphism in the type algebra}{S060}

\newcommand{\id}{\mathrm{id}}
\newnotation{id}{$\id$}{Identity homomorphism}{S061}

\newcommand{\ladj}[1]{\overleftarrow{#1}}
\newnotation{ladj}{$\protect\ladj{\morph}$}{Left adjoint of a type homomorphism}{S070}

\newcommand{\radj}[1]{\overrightarrow{#1}}
\newnotation{radj}{$\protect\radj{\morph}$}{Right adjoint of a type homomorphism}{S075}

\newcommand{\ladjrest}[1]{\underline{\overleftarrow{#1}}}
\newnotation{ladjrest}{$\protect\ladjrest{\morph}$}{Remainder of a left adjoint of a type homomorphism}{S080}

\newcommand{\radjrest}[1]{\underline{\overrightarrow{#1}}}
\newnotation{radjrest}{$\protect\radjrest{\morph}$}{Remainder of a right adjoint of a type homomorphism}{S085}


%%%%%%%%%%%%%%%%%%%%%%%%%%%%%%%%%%%%%%%%%%%%%%%%%%%%%%%%%%%%%%%%%%%%%%%%
% Constraints - algebraic subtyping
%%%%%%%%%%%%%%%%%%%%%%%%%%%%%%%%%%%%%%%%%%%%%%%%%%%%%%%%%%%%%%%%%%%%%%%%

\newnotation{constr}{$c$}{Constraint}{T03}

\newcommand{\cstr}[3]{\left\langle\!\left\langle #1 \vdash #2 : #3 \right\rangle\!\right\rangle}
\newnotation{cstr}{$\cstr{\Gamma}{e}{\tau}$}{Constraint generation for $\Gamma \vdash e : \tau$}{T04}

\newcommand{\equivct}{E}
\newcommand{\equivctx}{\ctx{\equivct}{\ctxhole}}
\newnotation{equivctx}{$\equivctx$}{Type equivalence context}{T07}

\newcommand{\solvect}{C}
\newcommand{\solvectx}{\ctx{\solvect}{\ctxhole}}
\newnotation{solvectx}{$\solvectx$}{Constraint solving context}{T08}

\newcommand{\undsym}[0]{\&}
\newcommand{\und}[0]{\;\undsym\;}
\newnotation{und}{$\und$}{Conjunction of constraints}{T10}

\newcommand{\tru}[0]{\mathbf T}
\newnotation{tru}{$\tru$}{Always-true constraint}{T20}

\newcommand{\fals}[0]{\mathbf F}
\newnotation{fals}{$\fals$}{Always-false constraint}{T25}

\newcommand{\cstreq}[0]{\cong}
\newnotation{cstreq}{$\cstreq$}{Constraint equivalence under the Boolean algebra}{T29}

\newcommand{\stepsto}[0]{\to}
\newnotation{stepsto}{$\stepsto$}{Solver step}{T30}

\newnotation{solved}{$\mathcal S(c)$}{Constraint in solved form}{T31}



% \newcommand{\progressesto}[0]{\rightsquigarrow}
% \newnotation{progressesto}{$\progressesto$}{Solver progress}{T33}

% \newcommand{\goesto}[0]{\rightarrowtriangle}
% \newnotation{goesto}{$\goesto$}{Solver return}{T36}

%%%%%%%%%%%%%%%%%%%%%%%%%%%%%%%%%%%%%%%%%%%%%%%%%%%%%%%%%%%%%%%%%%%%%%%%
% Fabric
%%%%%%%%%%%%%%%%%%%%%%%%%%%%%%%%%%%%%%%%%%%%%%%%%%%%%%%%%%%%%%%%%%%%%%%%

\newnotation{fbpat}{$p$}{Pattern}{Y009}

\newcommand{\fbrestricted}{\setminus}
\newnotation{fbcast}{$e \fbrestricted \ell$}{Record restriction -- removing a field $\ell$ from $e$}{Y010}

\newcommand{\fbcheck}[2]{{#1}_?{#2}}
\newnotation{fbcheck}{$\fbcheck{\ell}{e}$}{Checked projection -- a tagged value $\mathrm{Some}\,e{.}\ell$ if $\ell$ is in $e$, $\mathrm{None}\,()$ otherwise}{Y011}

\newcommand{\fbopt}[1]{\boxed{#1}?}
\newnotation{fbopt}{$\fbopt{\tau}$}{Optional field -- absent, or present with $\tau$}{Y012}

\newnotation{fbcase}{$\kappa$}{Case type}{Y019}

\newcommand{\fbtag}[2]{#1\,#2}
\newnotation{fbtag}{$\fbtag{T}{e}$}{Tagging}{Y020}

\newcommand{\fbuntag}[1]{!#1}
\newnotation{fbuntag}{$\fbuntag{e}$}{Untagging}{Y021}

\newcommand{\nom}{t}
\newnotation{nom}{$\nom$}{Nominal type}{Y030}

\newcommand{\fbcast}[3]{{#1} : {#2} \blacktriangleright {#3}}
\newnotation{fbcast}{$\fbcast{e}{\tau}{\tau}$}{Abbreviation-aware type cast}{Y031}

%%%%%%%%%%%%%%%%%%%%%%%%%%%%%%%%%%%%%%%%%%%%%%%%%%%%%%%%%%%%%%%%%%%%%%%%
% Star
%%%%%%%%%%%%%%%%%%%%%%%%%%%%%%%%%%%%%%%%%%%%%%%%%%%%%%%%%%%%%%%%%%%%%%%%

\newcommand{\exprcolor}[1]{#1}
\newcommand{\shapeexprcolor}[1]{#1}
\newcommand{\typecolor}[1]{#1}
\newcommand{\shapetypecolor}[1]{#1}

% \newcommand{\exprcolor}[1]{\mathcolor{RawSienna}{#1}}
% \newcommand{\shapeexprcolor}[1]{\mathcolor{BurntOrange}{#1}}
% \newcommand{\typecolor}[1]{\mathcolor{RoyalPurple}{#1}}
% \newcommand{\shapetypecolor}[1]{\mathcolor{Violet}{#1}}

\newnotation{ssexpr}{$e^S$}{Shape expression}{Z010}
\newnotation{ssval}{$s$}{Shape value}{Z011}
\newnotation{sstype}{$\sigma$}{Shape type}{Z012}

% - types
\newcommand{\sint}[0]{\typecolor{\mathrm{int}}}
\newnotation{sint}{$\sint$}{Integer type}{Z021}

\newcommand{\sfloat}[0]{\typecolor{\mathrm{float}}}
\newnotation{sfloat}{$\sfloat$}{Floating point type}{Z021}

\newnotation{sintv}{$n, m$}{Integer}{Z022}
\newnotation{sfloatv}{$f$}{Float}{Z022}

\newcommand{\sfntype}{\typecolor{\to}}
\newcommand{\fn}{\sfntype}
\newcommand{\srecordtype}[1]{\typecolor{\left\{{#1}\right\}}}
\newcommand{\svarianttype}[1]{\typecolor{\left[{#1}\right]}}

\newnotation{iota}{$\iota$}{Index for shape: upper bound on index type given shape type}{Z023}

\newcommand{\stwobounds}[2]{\typecolor{\left[ {#1}{.}{.}{#2} \right]}}
\newcommand{\sonebound}[1]{\typecolor{\left[ {#1} \right]}}
\newcommand{\sarraytype}[3]{\stwobounds{#1}{#2}{#3}}
\newcommand{\suniarraytype}[2]{\sonebound{#1}{#2}}
\newnotation{sfloat}{$\sarraytype{\sigma_1}{\sigma_2}{\tau}$}{Array type}{Z023}
\newnotation{sfloat}{$\suniarraytype{\sigma}{\tau}$}{Abbreviated array type ($\sarraytype{\sigma}{\sigma}{\tau}$)}{Z024}

% - expressions
\newcommand{\slam}[2]{\exprcolor{\lambda\,{#1} \ldotp {#2}}}
\newcommand{\slet}[3]{\exprcolor{\mathrm{let}\,#1=#2\,\mathrm{in}\,{#3}}}
\newcommand{\srecordval}[1]{\exprcolor{\left\{{#1}\right\}}}
\newcommand{\sarrayval}[2]{\exprcolor{\textsf{Arr}(#1,#2)}}
\newcommand{\sproj}[2]{\exprcolor{{#1}{.}{#2}}}
\newcommand{\stag}[2]{\exprcolor{{#1}\,{#2}}}
\newcommand{\smatch}[2]{\exprcolor{\mathrm{match}\,{#1}\,\mathrm{with}\,{#2}}}
\newcommand{\sbuild}[3]{\exprcolor{\Phi\, {#1}\left[{#2}\right] \ldotp {#3}}}
\newcommand{\sbuildempty}[2]{\sbuild{#1}{}{#2}}
\newcommand{\sindex}[2]{\exprcolor{{#1}\left[{#2}\right]}}
\newcommand{\sshape}[1]{\exprcolor{\left|{#1}\right|}}
\newcommand{\smult}{\times}

\newnotation{sarrayelemse}{$J$}{Unevaluated array elements}{Z030}
\newnotation{sarrayelemsv}{$I$}{Evaluated array elements}{Z031}
\newnotation{sarrayval}{$\sarrayval{s}{J}$}{Array literal expression}{Z032}
\newnotation{sarrayval}{$\sarrayval{s}{I}$}{Array literal value}{Z032}
\newnotation{sbuild}{$\sbuild{x}{e}{e'}$}{Array comprehension -- array of shape $e$ with element $e'$ at index $x$}{Z040}
\newnotation{sshape}{$\sshape{e}$}{Array shape access}{Z041}

% - Star shape values [we now better distinguish shape values from shape types]
% Shape values:
\newcommand{\eess}{{e^\mathrm{S}}}  %% was e_s
% \newcommand{\vess}{{v^\mathrm{S}}}  %% was v_s
\newcommand{\vess}{s}
\newcommand{\ssize}[1]{\shapeexprcolor{\#{#1}}}
\newcommand{\sproductval}[1]{\shapeexprcolor{\left\{\hspace{-2.7pt}\left|{#1}\right|\hspace{-2.7pt}\right\}}}
\newcommand{\sconcatval}[1]{\shapeexprcolor{\left\llbracket{#1}\right\rrbracket}}
\newcommand{\sbroadcast}[2]{\shapeexprcolor{{#1} \broadcast {#2}}}
\newcommand{\sproductproj}[2]{\shapeexprcolor{{{#1}\!\!\phantom{\mid}_{\circ}}#2}}
\newcommand{\sconcatproj}[2]{\shapeexprcolor{{{#1}\!\!\phantom{\mid}_{\diamond}}#2}}

\newnotation{ssize}{$\ssize{e}$}{Sized shape}{Z050}
\newnotation{sproduct}{$\sproductval{\cdots}$}{Product shape}{Z051}
\newnotation{sconcat}{$\sconcatval{\cdots}$}{Concatenation shape}{Z052}
\newnotation{sproductproj}{$\sproductproj{e}{\ell}$}{Product dimension projection}{Z053}
\newnotation{sconcatproj}{$\sconcatproj{e}{T}$}{Concatenation component projection}{Z054}

\newcommand{\broadcast}{\sqcap}
\newcommand{\broadcastsem}{\bowtie}
\newcommand{\scast}[2]{{#1} \obslash {#2}}
\newcommand{\inbs}{\vartriangleleft}
\newcommand{\inbsp}{\trianglelefteq}
\newcommand{\inbstr}{\blacktriangleleft}

\newnotation{broadcast}{$e \broadcast e$}{Broadcasting (expression)}{Z060}
\newnotation{broadcastsem}{$s \broadcastsem s$}{Broadcasting (semantics)}{Z061}
\newnotation{scast}{$\scast{v}{s}$}{Index coercion into shape}{Z062}
\newnotation{inbs}{$v \inbs s$}{Index in-bounds of shape}{Z063}
\newnotation{inbsp}{$v \inbsp s$}{Index structurally-exactly in-bounds of shape}{Z064}
\newnotation{inbstr}{$v \inbstr s$}{Index structurally-in-bounds of shape}{Z065}

% Shape types
\newcommand{\ssized}{\shapetypecolor{\#}}
\newcommand{\sproducttype}[1]{\shapetypecolor{\left\{\hspace{-2.7pt}\left|{#1}\right|\hspace{-2.7pt}\right\}}}
\newcommand{\sconcattype}[1]{\shapetypecolor{\left\llbracket{#1}\right\rrbracket}}

\newcommand{\evalct}{C}
\newcommand{\evalctx}{\ctx{\evalct}{\ctxhole}}
\newnotation{evalctx}{$\evalctx$}{Evaluation context}{Z070}

\newcommand{\step}{\leadsto}
\newcommand{\sterr}{\leadsto \lightning}
\newnotation{step}{$e \step e$}{Small-step (operational semantics)}{Z071}
\newnotation{sterr}{$e \sterr$}{Raises-error step (operational semantics)}{Z072}
\newnotation{refltransclos}{$R^*$}{Reflexive-transitive closure of relation $R$ (e.g.\@ $\step^*$, $\sterr^*$)}{Z073}


\frontmatter
  \begin{titlepage}
  \vspace*{5cm}
  \makeatletter
  \begin{center}
    \begin{Huge}
      \@title
    \end{Huge}\\[0.1cm]
    %
    \begin{Large}
      \@subtitle
    \end{Large}\\[1cm]
    \@author
    %
    \vfill
    A document submitted in partial fulfillment
    of the requirements for the degree of\\
    \emph{Master of Engineering in Computer Science}\\
    at\\
    \textsc{University of Cambridge}
  \end{center}
  \makeatother
\end{titlepage}

\newpage
\null
\thispagestyle{empty}
\newpage
  % \begin{center}
  \textsc{Abstract}
\end{center}
%
\noindent
%
Ducks are great. Let us have more of those!


  {\hypersetup{hidelinks}
  \tableofcontents}

%TC:endignore 
\mainmatter

  \chapter{Introduction}
\label{introduction}

Programs written in dynamically typed languages form a large fraction of modern programming. 
For many programmers, they are they easier to program in and enable more rapid development.
By leaving the programmer unrestricted by a static type system, dynamic languages are more \textbf{expressive} -- admitting more programs, many of which are useful, despite their lack of statically known types -- and thus more flexible, though also less safe.

Furthermore, for many domains -- such as array programming or data science, both of which are relevant to machine learning -- it has become a commonplace assumption that static type systems are too restrictive and get in the way of the programmer. Hence, programming in these domains is virtually always done without static types.
Thus, dynamic typing has become not only a matter of preference for higher expressiveness at the cost of safety, but also a \textbf{convention} in some domains.

\section{Strategy}

The benefits of static typing are clear, and researching approaches to bestowing static typing on otherwise dynamically typed programs is a useful problem to resolve -- proven in practice by the success of TypeScript (JavaScript's typed dialect) or Python's optional typing system. 
Obviously, not all dynamically typed programs can admit static typing under a reasonably complex type system. 
Hence, we must consider what \textbf{characteristics} of dynamically typed programs we are interested in capturing, and what \textbf{properties} we desire from our static type system. 

Given these desiderata, we can determine an \textbf{approach} for statically typing a reasonably large subset of programs with the target characteristics.
Having found such an approach, we can then inform the design of a statically typed language. 
Since we derived it by statically typing dynamically typed programs, we retain the expressivity of dynamic languages. However, in contrast to them, we can statically guarantee safety of the entire language.
We get the best of both worlds: a sliver of the expressiveness of dynamic languages with a healthy dose of statically ensured safety.

The design of a new language is not in itself useful, but it productively informs the evolution and extension of existing languages.
By restricting ourselves to solutions which work within the scope of our approach to static typing, we can also attempt to identify novel approaches to programming in domains which are traditionally untyped, upending existing conventions for dynamic typing. 

\section{Overview}

The thesis follows the strategy proposed in the previous section, and is split into several chapters. Each chapter offers contributions in different areas.

\subsubsection{Structural Subtyping: The Static Soul of Dynamic Languages.} 

I identify \textbf{duck typing} as a powerful characteristic pattern applied in dynamically typed programs. I propose \textbf{structural subtyping} as the mechanism for modelling duck typing statically. 

In need of a good model for a dynamic language, I introduce \textbf{Featherweight Lua} (FL) -- a simple $\lambda$ calculus with extensible records, following the tradition of object calculi \cite{abadi-cardelli-object-calculus}. I then give it a static type system with structural subtyping, capturing a \emph{well-behaved} subset of FL.

To show that FL remains adequately powerful when typed statically, I develop a \textbf{translation} from the well-known \textbf{Featherweight Java} (FJ) into FL. 
I thus formally show the intuitive notion that we can obtain a structurally typed language by erasing type definitions from a nominally typed one.
Indeed, the FJ-FL translation preserves well-typedness -- showing that the statically typed fragment of FL is at least as expressive as FJ, but without requiring any type definitions. 

To truly model a dynamic language, we need to rid programs of type annotations. This naturally leads to the question of providing \emph{type inference} for FL -- and generally, any language with a type system relying on structural subtyping -- which I address in the next chapter.

\subsubsection{Constraint-Based Algebraic Subtyping}

By using the recently invented \textbf{algebraic subtyping} technique, I develop a general \textbf{type inference framework} for languages with structural subtyping. 
The framework builds on the state-of-the-art work on algebraic subtyping -- enabling \textit{ML-style type inference}, combining parametric polymorphism with subtyping. 

Furthermore, I show an extension of algebraic subtyping with \textbf{type lattice homomorphisms}. Using them, I show type inference for extensible records -- as present in FL -- without the usual need for extending the type system with row polymorphism.

The description of the type inference approach is \emph{independent} of specific expression and type languages. Instead, it is given in terms of a generic \textbf{constraint language} and \textbf{solver} and some requirements. 

With this framework in hand, I can freely take advantage of its flexibility by designing a new language.

\subsubsection{Design and Implementation of Fabric}
I present my \textbf{design} of a functional language with structural subtyping -- \textbf{Fabric}. 
I consider case studies in which I show how Fabric can express patterns naturally possible in dynamically typed languages, but difficult without structural subtyping. 

I also investigate Fabric's \textbf{implementation} as a general-purpose programming language featuring structural subtyping and describe my compiler for it. In particular, I consider the implications of supporting structural subtyping on code generation -- for which I target WebAssembly.

As part of my Fabric compiler, I give a prototype implementation of type inference described in the previous chapter. It features a dedicated constraint generation routine for Fabric (and thus FL) -- showcasing that the type inference approach is indeed independent of the specific language specification.

\subsubsection{Structuring Arrays with Algebraic Shapes}
Inspired by the use of structural subtyping, I propose a novel statically typed calculus for array programming -- \textbf{Star}. 

The standard in practical array programming has been to forgo types, with dependent type systems proposed as virtually the only options for static typing. I instead propose a middle-ground: rich structural types for array shapes, helping the programmer build abstractions and avoid menial index arithmetic. At the same time, Star also admits type inference using my framework.

This chapter is based on a paper of the same name, which was written as part of the work on my thesis and coauthored with my supervisors. It was accepted for publication in the Proceedings of the 11th ACM ARRAY Workshop. 
  \chapter
    [Structural Subtyping: The Static Soul of Dynamic Languages]
    {Structural Subtyping: \newline The Static Soul of Dynamic Languages}
\label{static-soul}

This chapter both builds the motivation of this thesis, and introduces various key concepts -- such as nominal or structural (sub)typing -- that arise throughout the report.

We begin by exploring characteristics of dynamically typed programs that we might like to type statically (Section \ref{sec:ch2background}). I~propose for \textbf{duck typing} to be the key characteristic we should focus our attention on, and argue that \textbf{structural subtyping} is a solid approach for statically modelling this pattern. Afterwards, via the introduction of a \emph{translation} in Section \ref{sec:translations}, I ground intuitions about structural subtyping and its expressivity.

\section{Background}
\label{sec:ch2background}

We first consider the meaning of \emph{dynamic} and \emph{static} typing.

\emph{Dynamic typing} generally means any form of runtime type checking, while \emph{static typing} is type checking at compile time. A dynamically typed language (or program) relies only on dynamic typing, and not static typing. Since when we talk about \emph{typing} we generally mean static typing, we also call dynamically typed programs \emph{untyped}.
% \todo[color=red]{really untyped?} 

Many statically typed languages possess some facilities for dynamic type checking, blurring the line between static and dynamic languages: \begin{description}
    \item[Static can be dynamic] Java (along with many other object-oriented languages) features \textit{downcasts}, which coerce an object of some class into its subclass. This has to be checked at runtime to preserve safety. 
    % It would be entirely possible -- though impractical -- to program in Java using only the \texttt{Object} type and to rely on downcasts for any useful work, essentially circumventing static typing. 
    More glaringly, Java also has reflection facilities, allowing introspection of types of values at runtime. 
    \item[Dynamic can be static] Most dynamically typed languages \emph{could} be given a trivial static type system -- for example, where all well-formed expressions are just given some type $\star$. 
\end{description} 

Furthermore, the line between type checking and other kinds of checks is itself subjective.
Verifying that an index into an array is an integer is obviously known to be type checking.
On the other hand, determining whether the index is out-of-bounds is generally thought not to be a type check -- even though it is handled as such in a dependent type system. 

Hence, it is difficult say what dynamic language patterns we should attempt to type statically, and what type safety properties we should ensure. It is necessary to make some assumptions in order to proceed further.\todo{revise}
After all, the space of dynamically typed programs is extremely large -- but there are only so many interesting patterns hiding within.

\needspace{6em}
\subsection{Taming runtime type checking}
\label{subsec:runtime-type-checking}

Let us distinguish two kinds of runtime type checking -- with the goal of identifying a well-behaving subset: \begin{description}
    \item[Types as data] A value's type can be read and itself operated on as a value unique to the type. For example, both Python and Lua feature a \texttt{type} built-in function. Since the type becomes a runtime value, it can arbitrarily affect the data and control flow. 
    Branching on the type (e.g.\@ Python's \texttt{isinstance}) -- a common pattern in literature on \emph{set-theoretic types} -- also treats the type as data. 
    
    I exclude the principle of types as data, as it has the following conseqeuences: \begin{description}
        \item[Lack of \emph{parametricity}] -- accessing the type of a value is always legal, i.e. it is of type $\forall \alpha \ldotp \alpha \to \textsf{Type}$. However, since the type directly impacts the result of the computation, \emph{parametricity} no longer holds: it is no longer the case a parametrically-polymorphic function performs the same computation in any type instantiation. This stops us from enjoying useful properties -- like theorems for free \cite{theorems-for-free} -- and leads to the second point.
        \item[Lack of \emph{predictability}] -- Subjectively speaking, code using types as data is more complicated and confusing to follow. This is reflected in any attempt to statically analyse them -- even when we limit ourselves to branching on types, it is known that we are limited computationally by the need for \emph{backtracking} \cite{polymorphic-set-theoretic-types}.
    \end{description}
    \item[Types as runtime guardrails] A value's type could instead be consulted only when we perform an operation on it, in order to check whether the operation is legal -- with an error raised otherwise -- similarly to how types are used in statically typed languages.     
    
    For example, consider a \emph{record} (or \emph{object}) data type, which stores some list of labelled fields, and is commonly featured in dynamic languages. 
    We usually consider the labels of fields present to be part of the record's type.
    Accessing a field of an object requires a (dynamic \emph{or} static) type check for whether it is present in the object.
    
    Viewing types as guardrails -- sources of merely runtime type errors, and not information about a value -- follows the \textbf{duck typing} pattern \cite{duck-typing} of untyped programs. If we expect an object to \texttt{quack()}, then we worry about nothing else but for our program to \texttt{quack()}. This embodies the principle \textit{\enquote{ask for forgiveness, not permission}}.
    % -- programs assume that all operations are legal.
    Note that -- provided we do not handle errors raised due to illegal operations -- duck typing does not suffer from the same loss of parametricity.
\end{description}
I chose to design an approach to static typing which admits \textbf{duck-typed} programs -- motivated by preservation of parametricity. 
We now have to find a way to statically model duck typing.

\subsection{Modelling duck typing}
\label{subsec:duck-models}

Having identified duck typing as a crucial well-behaved pattern in dynamically-typed programs, it remains to find its \emph{static soul} -- a method which admits a corresponding pattern, but can be statically typed.

\subsubsection{Structural subtyping}

Duck typing leads us to a natural notion of \textbf{subtyping}: if a duck $a$ of type $A$ quacks, and another duck $b$ of type $B$ not only quacks but also walks -- then clearly $b$ can be used in any place $a$ can. A substitution principle holds: the type $B$ supports more operations than $A$, and thus $B$ is a subtype of $A$ -- we denote this $B \sub A$. When speaking of subtyping, we usually refer to its \emph{implicit} variety -- the language does not require an \emph{explicit} annotation every time we treat a type as its supertype.

\needspace{1em}
This is a good time to introduce a distinction between \textbf{nominal} and \textbf{structural} (static) typing \cite{tapl}: \begin{description}
    \item[Nominal] Most static type systems in popular programming languages are nominal: types are introduced via a type definition, where they are \emph{named} -- the type is then identified with (e.g.\@ compared by) this name. For example, the type systems in Java, C/C++, and Haskell are predominantly nominal.
    \item[Structural] On the other hand, a structural type system does not pose the requirement for a type to have a name nor a definition. A popular example of a structurally typed language would be TypeScript -- the statically typed dialect of JavaScript. 
\end{description}
Similarly to static and dynamic, nominal and structural typing also lie on a spectrum: few languages feature solely nominal or structural types. A good example of a language with a mix of nominal and structural typing is OCaml (Figure \ref{fig:nominal-and-structural-ocaml}): while records and variants can only be introduced in a \emph{nominal} type definition, we can introduce \emph{structural} type abbreviations. In particular, we have structural versions of records (objects) and variants (polymorphic variants). OCaml modules are structurally typed, too.

\begin{figure}
    \centering
    \begin{subfigure}{.49\textwidth}
    \centering
    \begin{ocaml}
(* record *)
type point = { x : int; y : int }
(* variant *)
type opt = None | Some of int
    \end{ocaml}
    \caption{Nominally typed records and variants.}
    \label{subfig:nominal-ocaml}
    \end{subfigure}
    \hfill
    \begin{subfigure}{.49\textwidth}
    \centering
    \begin{ocaml}
(* object *)
type point = < x : int; y : int >
(* polymorphic variant *)
type opt = [ `None | `Some of int ]
    \end{ocaml}
    \caption{Structurally typed objects and polymorphic variants.}
    \label{subfig:structural-ocaml}
    \end{subfigure}
    \caption{Examples of nominal and structural type definitions in OCaml. Note that the same syntax is used for both nominal type definitions (Figure \ref{subfig:nominal-ocaml}) and structural type abbreviations (Figure \ref{subfig:structural-ocaml}), even though these two kinds of type declarations behave differently in OCaml's type system.}
    \label{fig:nominal-and-structural-ocaml}
\end{figure}

Intuitively, a nominally typed program can be transformed into a structurally typed one, and as such structurally typed programs are inherently more flexible -- we explore this in Section \ref{sec:translations}. 
% This fundamental idea of \emph{anonymising types} by erasing their names is formally explored in the \textbf{translations} presented in Section \ref{sec:translations} on the languages we introduce in Section \ref{sec:languages}.

We also introduce a similar distinction between nominal and structural \textbf{subtyping}: while in a language like Java subtyping is defined through the inheritance hierarchy (\fj{class C extends C}'\fj{ \{...\}}) -- on types with specific \emph{names} -- structural subtyping is instead defined in terms of the structure of the type. Structural subtyping naturally arises on records (and, similarly, variants) through the following two rules:\footnote{A traditional formal treatment is given by \textcite{tapl}. In this dissertation -- to express it more conveniently for algebraic subtyping in Chapter \ref{algebraic-subtyping} -- we use a slightly different approach with \emph{field types}.}
\begin{description}
    \item[Width subtyping] A record is a subtype of another if it has more fields and the other are compatible, e.g.: $$ \{ \mathrm{foo} : \mathrm{int}, \mathrm{bar} : \mathrm{string} \} \sub \{ \mathrm{foo} : \mathrm{int} \} $$
    \item[Depth subtyping] A record is a subtype of another if its respective fields are subtypes, e.g.:
    $$ \{ \mathrm{foo}: \mathrm{nat}, \mathrm{bar}: \mathrm{string} \} \sub \{ \mathrm{foo}: \mathrm{int}, \mathrm{bar}: \mathrm{string} \} $$
\end{description}

We formalise this in the type system for the record calculus -- Featherweight Lua -- in Section~\ref{subsec:featherweight-lua}.

Clearly, \textbf{structural subtyping} is most relevant for modelling duck typing: most dynamically typed programs do not feature type definitions.\footnote{A note-worthy exception would be OOP-style class definitions, present in e.g.\@ Python.} I am mainly motivated by modelling objects in dynamic languages, following the tradition of \textcite{cardelli-multiple-inheritance}.

Note that we use the name structural subtyping in the sense of subtyping in a structurally typed setting, like \textcite{dolan-thesis} or \textcite{cardelli-power-type} -- and not in the sense of the non-structural and structural split explored by e.g.\@ \textcite{subtyping-decidability}.

I chose structural subtyping as the direction for my thesis -- we briefly consider two possible alternatives in the following subsections. I motivate this choice mainly by the recency -- relative to the plethora of work on structural subtyping from the 90s -- of the seminal work of \textcite{mlsub} on \emph{algebraic subtyping}, as it is directly applicable to languages with structural subtyping. It is thus the topic of Chapter \ref{algebraic-subtyping}.

\subsubsection{Alternative 1: Row polymorphism}

\begin{figure}
    \centering
    \begin{tabular}{c}
    \begin{ocaml}
let f = function 
    | `None -> `Unit 
    | `Some x when x mod 2 = 0 -> `Pair (x / 2, x / 2) 
    | `Some x -> `Single x
(* 
val f : [< `None | `Some of int ] 
     -> [> `Pair of int * int | `Single of int | `Unit ]
*)
    \end{ocaml}
    \end{tabular}
    \caption{Example of row polymorphism in OCaml using polymorphic variants, annotated with its (inferred) most general type. \texttt{<} and \texttt{>} stand for row type variables in closed and open variants, respectively. The argument type can be unified against any variant with at most the listed cases, while the result with at least those.}
    \label{fig:ocaml-row-polymorphism}
\end{figure}

There is a common folklore trick for replacing subtype polymorphism -- as seen for objects/records above -- by an appropriate form of parametric polymorphism \cite{structural-subtyping-as-parameric-polymorphism}. For example, a function of type $\top \to \mathrm{int}$ -- where $\top$ is top, the supertype of any type -- could equivalently be given the type scheme $\forall \alpha \ldotp \alpha \to \mathrm{int}$, since $\alpha$ can be instantiated to any argument type.
\textbf{Row polymorphism} \cite{remy-records}, introduced by \textcite{wand-rows} (more generally \emph{structural polymorphism} \cite{simple-structural-polymorphism}) -- can be seen as an application of this trick to structural record and variant types: we introduce \emph{row type variables} that stand for \enquote{the rest of the record}.
% Width subtyping correspond to instantiation of row variables, while depth subtyping corresponds to nested instantations.

% https://ocaml.org/papers
% https://brianmckenna.org/blog/row_polymorphism_isnt_subtyping
A technical explanation of row polymorphism is outside the scope of this thesis, so we focus on the practical example of OCaml's objects \cite{objective-ml} and polymorphic variants \cite{polymorphic-variants} -- as exemplified in Figure \ref{subfig:structural-ocaml}. These are typed using row polymorphism with \emph{implicit} row variables \cite{objective-ml}, which can cause some awkward limitations \cite{castagna-polymorphic-variants}. OCaml also supports \emph{explicit} (super)type coercions anyhow, adding a form of explicit structural subtyping -- managing the limitations of the row polymorphism design. 
An example application of row polymorphism in OCaml is given in Figure \ref{fig:ocaml-row-polymorphism}.

The relative power of structural subtyping and row polymorphism -- as extensions on parametric polymorphism -- is unclear.
% , and building systems based on row polymorphism is a viable alternative -- though possibly limited in the context of types for which rows are not clearly helpful in modelling notions of subtyping. 
There is some recent work exploring the relationship between the two \cite{disjoint-polymorphism, structural-subtyping-as-parameric-polymorphism}. However, systems with row polymorphism tend to grow more complex than ones with subtyping \cite{castagna-polymorphic-variants}. Nonetheless, they have historically been used as a common extension of ML-like type systems.

\subsubsection{Alternative 2: Set-theoretic types}

With \textbf{set-theoretic types}, we extend the type language with operators corresponding to set operations:
$$ \tau ::= \cdots \mid \tau \cup \tau \mid \tau \cap \tau \mid \lnot \tau $$
These operators admit a set-theoretic interpretation -- they are exactly compatible with the sets of expressions (values) that have a given type $\tau$. 
Under this interpretation, the set-inclusion relation naturally gives us a notion of subtyping, so that e.g.\@ $\tau \cap \tau' \sub \tau'$.
Set-theoretic types are elegant, expressive, and have seen practical use \cite{set-theoretic-types-for-elixir, set-theoretic-types-for-erlang}. However, it is difficult to efficiently infer and simplify them, and type inference usually lacks \emph{principal types}, enjoyed by ML-style type inference (algebraic subtyping included \cite{mlstruct, castagna-dynamic}). 

Later, in algebraic subtyping\todo[color=green]{move?} -- a technique that builds on structural subtyping -- we rely on a type lattice construction featuring meets $\meet$ (least upper bounds) and joins $\join$ (greatest lower bounds). Though similar at first sight, the two approaches are subtly different \cite{mlstruct}: while $(\mathrm{int} \to \mathrm{int}) \cup (\mathrm{nat} \to \mathrm{nat})$ is as valid a type as any other, in the type lattice of MLsub \cite{mlsub} and MLstruct \cite{mlstruct} we have: $$(\mathrm{int} \to \mathrm{int}) \join (\mathrm{nat} \to \mathrm{nat}) = (\mathrm{int} \meet \mathrm{nat} \to \mathrm{int} \join \mathrm{nat}) = \mathrm{nat} \to \mathrm{int} $$
While $\cup$ gives us a function type which preserves its argument's type, $\join$ only grants that the function must accept $\mathrm{nat}$ and only returns $\mathrm{int}$ (cf.\@ subtyping of functions).
Hence, types no longer have a direct set-theoretic interpretation.\footnote{Note that type lattices exist where the equation does not hold, and that Dolan's system admits some set-theoretic-esque types like $\{\} \meet (\top \to \top)$ (though they can only be used as $\top$) where necessary for \emph{extensibility}.} It thus seems that set-theoretic types are at least as expressive as approaches based on algebraic subtyping, where types are constrained to just the ones in the defined lattice. To the author's knowledge, no formal relationship between the two has been established.

\section{Languages}
\label{sec:languages}

As I have now determined the main approach to static typing in this thesis -- \textbf{structural subtyping} -- we are in need of some initial model languages to ground any further conclusions.

Firstly, I construct a simple calculus, which I dub \textbf{Featherweight Lua} (FL), that we will use to model a dynamically typed language with a statically typed fragment featuring structural subtyping. As the name implies, FL admits a simple embedding into its namesake, the popular dynamically typed language Lua. Secondly, I recall the established \textbf{Featherweight Java} (FJ) calculus \cite{featherweight-java}, and use it as a model for a nominally typed OOP language.  

\subsection{Featherweight Lua}
\label{subsec:featherweight-lua}

I construct FL as the simply-typed lambda calculus extended with extensible record types. Expressions and types in FL are given in Figure \ref{fig:featherweight-lua-grammar}. We consider key aspects in the following subsections.

\begin{figure}
    \centering
    \begin{subfigure}{.49\textwidth}
\begin{align*}
    \graintro e
    x 
    & \text{(variable)}
    \graitem
    \fllet{x}{e}e
    & \text{(let-binding)}
    \graitem
    \fllam{x} e
    & \text{(function)}
    \graitem
    e\,e
    & \text{(application)}        
    \graitem
    \flrec{\overline{\ell = e}}
    & \text{(record construction)}
    \graitem
    \flext{\ell = e}{e}
    & \text{(record extension)}
    \graitem
    \flproj{e}{\ell}
    & \text{(record projection)}
    \graitem
    \flcast{e}{\tau}
    & \text{(subtype coercion)}
    \\ \\
    \graintro v
    \fllam{x} e
    & \text{(function)}
    \graitem
    \flrec{\ell = v}
    & \text{(record)}
\end{align*}
\caption{Expressions $e$ and values $v$ in Featherweight Lua.}
\label{fig:featherweight-lua-expr}
\end{subfigure}
\hfill
\begin{subfigure}{.49\textwidth}
\begin{align*}
    \graintro \tau
    \tau \to \tau
    & \text{(function)}
    \graitem
    \flext{\overline{\ell : \phi}}{\phi}
    & \text{(record)}
    \\ \\ 
    \graintro \phi 
    \top
    & \text{(top)}
    \graitem
    \bot 
    & \text{(bottom)}
    \graitem
    \tau
    & \text{(present)}
    \graitem
    \flabsent 
    & \text{(absent)}
\end{align*}
$$ \flrec{ \overline{\ell : \tau_\ell} } \defeq \flrec{ \overline{\ell : \tau_\ell} \mid \top } $$
\caption{Types $\tau$ and field types $\phi$.}
\label{fig:featherweight-lua-types}
\end{subfigure}

    \caption{Syntax of Featherweight Lua.}
    \label{fig:featherweight-lua-grammar}
\end{figure}

\subsubsection{Expressions}

Since FL's functions and bindings are entirely standard, we consider its records and coercions:\todo{opsem?}
\begin{description}
    \item[Records] Records can be constructed from a list of individual assignments, and individual fields can be projected. Since FL's records are extensible, we have a record extension operator, which, for an existing record lacking a given field, returns a new record with that field set.
    \item[Coercion] FL features a type coercion operator, which is a no-op at runtime. It is included as it plays a role in the FJ-FL translation later.
\end{description}

\subsubsection{Typing}

\begin{figure}
    \centering
    $$ 
\begin{array}{c}
\flproj{\tau}{\ell} \defeq \begin{cases}
    \phi_\ell, & \ell \text{ occurs in } \tau \\
    \phi, & \ell \text{ does not occur in } \tau
\end{cases} \qquad \text{where } \tau = \flext{ \overline{\ell : \phi_\ell}}{\phi}
\\[1.5em]
\boxed{\tau \sub \tau}
\\[0.5em]
\irule{FL-Sub-Fun}{\tau_1 \super \tau_2 \quad \tau_1' \sub \tau_2'}{\tau_1 \to \tau_1' \sub \tau_2 \to \tau_2'}
\qquad
\irule{FL-Sub-Rec}{\forall \ell \in \mathcal L(\tau_1) \cup \mathcal L(\tau_2) \ldotp \flproj{\tau_1}{\ell} \sub \flproj{\tau_2}{\ell}}{\tau_1 \sub \tau_2}
\\[2.5em]
\boxed{\phi \sub \phi}
\\[1em]
\flfbot \sub \tau \sub \flftop \qquad \flfbot \sub \flabsent \sub \flftop
\end{array} 
$$
    \caption{Declarative definition of the subtyping relation $\sub$ in Featherweight Lua, including on field types $\phi$. We define projection on record types,~$\flproj{\tau}{\ell} = \phi.$. Note that when deciding record subtyping we only need to check the defaults and occurring labels.}
    \label{fig:featherweight-lua-subtyping}
\end{figure}

\begin{figure}
    $$ 
\begin{array}{c}
\boxed{\Gamma \vdash e : \tau} 
\\[0.5em]
\Gamma ::= \cdot \mid \Gamma, x : \tau
\\[1em]
\irule{FL-Typ-Var}{\Gamma(x) = \tau}{\Gamma \vdash v : \tau}
\quad 
\irule{FL-Typ-Sub}{\Gamma \vdash e : \tau \quad \tau \sub \tau'}{\Gamma \vdash e : \tau'}
\quad
\irule{FL-Typ-Cast}{\Gamma \vdash e : \tau}{\Gamma \vdash \flcast e \tau' : \tau'}
\\[2em]
\irule{FL-Typ-Let}{\Gamma \vdash e : \tau \quad \Gamma, x : \tau \vdash e' : \tau'}{\Gamma \vdash \fllet{x}{e}{e'} : \tau'}
\quad 
\irule{FL-Typ-Fun}{\Gamma, x : \tau \vdash e : \tau'}{\Gamma \vdash \fllam{x} e : \tau \to \tau'}    
\quad 
\irule{FL-Typ-App}{\Gamma \vdash e : \tau' \to \tau \quad \Gamma \vdash e' : \tau'}{\Gamma \vdash e\,e' : \tau}
\\[2em]
\irule{FL-Typ-Cons}
    {\overline{\Gamma \vdash e_\ell : \tau_\ell}}
    {\Gamma \vdash \flrec{\overline{\ell = e_\ell}} : \flrec{\overline{\ell : \tau_\ell} \mid \flabsent}}
\quad
\irule{FL-Typ-Ext}
    {\Gamma \vdash e : \flext{\ell^\star : \flabsent, \overline{\ell : \phi_\ell}}{\phi_\ell'}}
    {\Gamma \vdash \flext{\ell^\star = e^\star}{e} : \flext{\ell^\star : \tau^\star, \overline{\ell : \phi_\ell}}{\phi_\ell'}}
\quad
\irule{FL-Typ-Proj}
    {\Gamma \vdash e : \flrec{ \ell : \tau }}
    {\Gamma \vdash e : \tau}
\end{array} 
$$
    \caption{Typing rules for the statically typed fragment of Featherweight Lua.}
    \label{fig:featherweight-lua-typing}
\end{figure}

We give the declarative definition of FL's subtyping in Figure \ref{fig:featherweight-lua-subtyping}, and typing rules in Figure \ref{fig:featherweight-lua-typing}.
FL has only two type constructors -- usual functions and slightly-unusual records. We consider some details.

\begin{description}
    \item[Recursive types] Types $\tau$ in FL are generated coinductively, so that we have regular (equi)recursive types (everything else generated inductively as usual). This later allows us to statically type (object-oriented) open recursion -- a method calling other methods of the same class and accessing its attributes.
    \item[Record field types] Instead of the usual width and depth subtyping rules for record subtyping, we instead consider record types $\flrec{\overline{\ell : \phi} \mid \phi'}$ as functions from labels $\ell$ into \emph{field types} $\phi$, which are unequal to the \emph{default} $\phi'$ at finitely many points, which are given by the list $\overline{\ell : \phi}$. This approach is convenient, as it generalises well to keeping track of absent fields (fillable by extension). It is also trivially compatible with the usual definition, like that of \textcite{tapl}, taking just present and top field types.
    \item[Record operations] A freshly constructed record has all its other fields absent by default, and projection does not constrain the rest of the record (it allows $\flftop$ fields). Note no record value has a $\flfbot$ field.
\end{description}

\subsubsection{Embedding in Lua}

\begin{figure}
    \centering
    \begin{align*}
\Aboxed{\denotl{e} &= \lua{e}} \\ 
\denotl{x} &= x \\ 
\denotl{\fllet{x}{e} e} &= x \lua{ = } \denotl{e} \lua{; } \denotl{e} \\
\denotl{\fllam{x} e} &= \lua{function (} x\lua{) return }\denotl{e}\lua{ end} \\ %
\denotl{e\,e} &= \denotl{e} \lua{(} e \lua{)} \\
\denotl{\flrec{\overline{\ell = e}}} &= \lua\{ \overline{\ell = \denot e} \lua\} \\
\denotl{\flext{\ell = e}{e'}} &= \lua{update(}\ell, e, e'\lua{)}  \\
\denotl{\flproj{e}{\ell}} &= \denotl{e}\lua{.}\ell \\
\denotl{\flcast{e}{\tau}} &= \denotl{e} \\
\centredalign
\end{align*}
\begin{minipage}{7.5cm}\lua{%
function update(l, e, t) \\
\hspace*{0.6cm}local target = \{\} \\
\hspace*{0.6cm}for k, v in pairs(t) do target[k] = v end \\
\hspace*{0.6cm}target[l] = e \\ 
end \\
}
\end{minipage}

    \caption{Embedding of Featherweight Lua expressions $e$ into Lua programs $\lua e$. The embedding makes use the use of the functional \textsf{update} for tables, defined in Lua code as part of a preamble. \\
    This is only a sketch of the translation -- for instance, any assignments resulting from let-bindings should be extracted as separate statements and sequenced with \enquote{;}, as Lua does not have binding expressions.}
    \label{fig:featherweight-lua-embedding}
\end{figure}

Featherweight Lua is, notionally, a subset of a widely known dynamically typed language: Lua \cite{lua54}.
Lua's core data structure is the \emph{table}, which we replicate -- though with strictly less power (without dynamic key access, Lua's ``metatables'', etc.) -- as FL's records. 
Lua also features first-class functions.
A key distinction between FL and Lua is the former's lack of mutability. 

To show the compatibility of FL with its namesake, I sketch a semantics-preserving compositional embedding of FL into Lua in Figure \ref{fig:featherweight-lua-embedding}.
% (there are no types to preserve for Lua).

\subsubsection{Untyped Featherweight Lua}

Generally, we use the name FL to talk of its statically typed part. We also have the dynamically typed (untyped) fragment, dubbed Untyped Featherweight Lua, which admits more potentially useful programs. 
The following expression $e_\mathrm{untyped}$ is not well-typed, but evaluates at both $e_\mathrm{cond} = \fllam x \fllam y x$ and $\fllam x \fllam y y$:
$$ e_\mathrm{untyped} = \fllet{x}{e_\mathrm{cond}\,\flrec{a=\flrec{}}\,\flrec{b=\flrec{}}} e_\mathrm{cond}\,(\fllam y \flproj{y}{a})\,(\fllam y y{.}b)\,x  $$
Reasoning about correctness is more difficult than in the well-typed case: it is the fact the same branch is taken by $e_\mathrm{cond}$ that leads to successful evaluation.

\subsection{Featherweight Java}
\label{subsec:featherweight-java}

\textcite{featherweight-java} introduced Featherweight Java: the \emph{core functional calculus} of Java. It contains only the key features of the languages -- \textbf{classes} and \textbf{objects} with \textbf{fields} and \textbf{methods} -- without mutation. To us, FJ is a nominally (statically) typed calculus with object-oriented features and subtype polymorphism. The cited paper offers the formal details. An example FJ program is in Figure \ref{fig:featherweight-java}.

\begin{figure}
    \centering
    \begin{tabular}{c}
\begin{java}
class Pair extends Object {
    Object fst;
    Object snd;
    Pair(Object fst, Object snd) {
        super(); this.fst = fst; this.snd = snd;
    }
    Pair setfst(Object newfst) {
        return new Pair(newfst, this.snd);
    }
    Pair setsnd(Object newsnd) {
        return new Pair(this.fst, newsnd);
    }
}
\end{java}
\end{tabular}

    \caption{Example Featherweight Java program implementing a $\fj{Pair}$ class, given by \textcite{featherweight-java}}
    \label{fig:featherweight-java}
\end{figure}

Considering structural subtyping, we use FJ in the following thought experiment: what if we \emph{erased} all the type names (FJ classes) in a nominally typed program, inlining their definitions? This process of \textbf{anonymising types} intuitively works -- and shows that nominally typed programs have within them a structurally typed heart. We focus on the case of languages with subtype polymorphism (FJ and FL) -- more complex cases (e.g.\@ like Haskell's higher-kinded polymorphism) can be difficult to combine with structural typing (cf.\@ structural subtyping of ML modules in \textcite{modular-implicits}).
Via the FL-FJ translation in Section \ref{sec:translations} we will see that, despite its rich classes and objects, FJ is no more expressive than FL.

\section{Anonymising types: FL-FJ translation}
\label{sec:translations}

To serve as groundwork for this thesis, I now build a formal understanding of the relationship between the nominal, structural, and dynamic typing disciplines. To this end, we consider a translation between the languages above -- from the nominally typed Featherweight Java into the structurally (statically) typed fragment of Featherweight Lua. 

The translation follows long-known relationships between object-oriented programming and lambda calculus with records (summarised by e.g.\@ \textcite{pierce-thesis}) -- but focuses on relating nominal and structural typing rather than semantics.\todo{cite \cite{intrinsic-extrinsic}}


\begin{figure}
    \centering
    \begin{align*}    
    \Aboxed{\denoty{\fj K} &= \tau} \\
    \denoty{\fjconsdef} &= \overline{\denoty{\fj C} \to {}} \denoty{\fj{C}} \\
    \\
    \Aboxed{\denoty{\fj M}_{\fj C} &= x : \tau} \\
    \denoty{\fjmethdefx} &= \fj m : \denoty{\fj C} \to \overline{\denoty{\fj C} \to {}} \denoty{\fj{C}^*} \\
    \\
    \Aboxed{\denoty{\fj L} &= x : \tau} \\
    \denoty{\fjclassdef} &= \fj C : \flrec{\overline{\fj{f}: \denoty{\fj{C}}}, \overline{\denoty{\fj M}}} \\
    \\
    \Aboxed{\denoty{\Gamma_{\fj C}} &= \Gamma_\tau} \\
    \denoty{\overline{\fj x : \fj C}} &= \overline{\fj x : \denoty{\fj C}} \\
    \\
    \Aboxed{\denot{\fj K}_{\fj{C}}^{\fj{C}'} &= e} \\
    \denot{\fjconsdef}_{\fj{C}}^{\fj{C}'} &= \flletx{\fj C}{\overline{\fllam{\fj f : \denoty{\fj C}}}\, \flext{\fj{\fjoverline{f' = f''}}, \fj C_\mathrm{proto}}{\fj{C'}\,\overline{\fj{f}^*}}}
    \\ \\
    \Aboxed{\denot{\fj M}_{\fj C} &= e} \\
    \denot{\fjmethdef}_{\fj C} &= \flletx{\fj m}{\fllam{\fj{this} : \denoty{\fj C}} \overline{\fllam{\fj x: \denoty{\fj C}}}\, \denot{\fj e}}
    \\ \\
    \Aboxed{\denot{\fj L} &= e} \\
    \denot{\fjclassdef} &= \flletx{\fj C}{\denot{\fj K}_{\fj C}^{\fj C'}} \\
    &\phantom{{}={}} \flletx{\fj{C}_\mathrm{proto}}{\flext{\overline{\denot{\fj M}_{\fj C}}}{\fj C'_\mathrm{proto}}} \\
    \fj{Object}_\mathrm{proto}\,r &= r
    \\ \\
    \Aboxed{\denot{\fj e} &= e} \\
    \denot{\fj x} &= \fj x \\
    \denot{\fj {e.f}} &= \flproj{\denot{\fj e}}{\fj f} \\
    \denot{\fj {e.m(\fjoverline e)}} &= \fllet{x}{\denot{\fj e}} \flproj{x}{\fj m}\,x\,\overline{\denot{\fj e}} \\
    \denot{\fj {new C(\fjoverline e)}} &= \fj{C}\,\overline{\denot{\fj e}} \\
    \denot{\fj {(C)e}} &= \flcast{\denot {\fj e}}{\denoty {\fj C}} 
    % \centredalign
\end{align*}
    \caption{The translation from Featherweight Java into the statically typed fragment of Featherweight Lua. We have translations for expressions $\denot{-} = e$ and types $\denoty{-} = \tau$. Translation of methods also preserves their name, so that they may be used as record labels. Translations of classes $\fj C$ make implicit use of the class table $\mathrm{CT}$, which is part of an FJ program. Since we know the list of fields $\fj C_\mathrm{proto}$, we can abuse notation and use it for extension (and do so in the translation of a constructor $K$).}
    \label{fig:translation}
\end{figure}

\subsection{Construction}

The translation is described formally in Figure \ref{fig:translation}, with some notational simplifications for brevity. The FJ notation is the same as in the original paper \cite{featherweight-java} -- its Java-like syntactic constructs should be familiar.

The translation follows directly by considering a \textbf{prototype-based} object-oriented programming style: we translate each class into a constructor $\fj C$ and the \emph{prototype} $\fj C_\mathrm{proto}$, holding all of the class's methods. Constructing a new class first calls the subclass constructor, and then extends the object with methods from the prototype and the class's fields.

A tricky aspect is that each method must take the object itself -- under the name $\fj{this}$ -- as an argument. Thankfully, method applications $\fj{e.m(\fjoverline e)}$ in FJ are syntactically separate, so we can pass the caller object as an argument to its translated method. 
We note that for this to be well-typed in FL, we require equirecursive types, so that we write down method types (which both occur in and contain the class type $\denoty{\fj C}$).

Lastly, note that while FL's type system only allows absent record fields to be extended, this condition is always satisfied -- FJ does not allow field shadowing nor method overriding.

\subsection{Correctness}

We now state and conjecture\todo{prove?} correctness criteria for the translation.

\begin{conjecture}[Preservation of typing]
    Typing of FJ expressions is preserved, so that if we have $\Gamma \vdash \fj e : \fj C$, then $\denoty{\Gamma} \vdash \denot{\fj e} : \denoty{\fj C}$. Furthermore, translations of FJ definitions yield FL terms with expected types:
    $$ \Gamma \vdash \denot{K}_{\fj C}^{\fj C'} : \denoty{K} \qquad \Gamma \vdash \denot{M}_{\fj C} : \denoty{M}_{\fj C} \qquad \Gamma \vdash \denot{L} : \denoty{L} $$
\end{conjecture}

\begin{conjecture}[Preservation of semantics]
    If we have $\cdot \vdash \fj e : \fj C$ and $\fj e \step^* \fj v$, then $\denot{\fj e} \step^* \denot{\fj v}$.
\end{conjecture}

In these statements, we implicitly assume a well-typed class table $\mathrm{CT}$ (as defined by \textcite{featherweight-java}) -- it is necessary both for types and semantics.

Note that, as pointed out by \textcite{inheritance-subtyping}, inheritance is \emph{not} subtyping -- it does not follow that $\denoty{\fj C} \sub \denoty {\fj C'}$ for any class $\fj C$ that $\fj{extends C}'$.

\begin{figure}[t]
    \centering
    \begin{tikzpicture}
        \node[align=center] at (-4,  0) (fj) {Featherweight \\ Java};
        \node[align=center] at ( 0,  0) (fl) {Featherweight \\ Lua};
        \node[align=center] at ( 4,  0) (ufl) {Untyped \\ Featherweight \\ Lua};
        \node at (-4, -2) (java) {Java};
        \node at ( 4, -2) (lua) {Lua};
        \node at (-4, 1.5) (nominal) {\emph{Nominal}};
        \node at ( 0, 1.5) (structural) {\emph{Structural}};
        \node at ( 4, 1.5) (dynamic) {\emph{Dynamic}};
        \draw[-latex, double] (fj) -- (fl) node[midway,above] {\footnotesize Fig. \ref{fig:translation}}; 
        \draw[-latex, double] (fj) -- (java) node[midway,right] {\footnotesize \cite{featherweight-java}};
        \draw[-latex] (fl) -- (ufl);
        \draw[-latex] (ufl) -- (lua) node[midway,left] {\footnotesize Fig. \ref{fig:featherweight-lua-embedding}};
        \draw[dashed] (-2, -2.5) -- (-2, 2);
        \draw[dashed] (2, -2.5) -- (2, 2);
    \end{tikzpicture}
    \caption{Diagram depicting a scale of nominal, structural, and dynamic typing, with languages connected by translations between them. Its core is the translation of Featherweight Java (nominal) into Featherweight Lua (structural) from Section \ref{sec:translations}. Double arrows $\Rightarrow$ are translations that preserve both semantics and (non-trivial) types, while single arrows $\to$ only preserve semantics.}
    \label{fig:nominal-structural-dynamic}
\end{figure}

\subsection{Consequences}

Existence of the FL-FJ translations shows that very simple structural subtyping is sufficient to capture the essence of classic object-oriented programming, as in (Featherweight) Java -- and its nominal approach to static typing.
More generally, we conclude that structural types are a more flexible typing discipline, as nominal types can be straightforwardly erased from simply-typed programs.

We contextualise this with the fact the structurally typed FL remains useful when we consider its untyped variety. Its statically typed fragment captures some -- but not all -- programs that do not go wrong.

We conclude that \textbf{structural subtyping brings us closer to dynamic languages}. In conjunction, we have placed nominal, structural, and dynamic typing on a scale -- a relationship depicted in Figure \ref{fig:nominal-structural-dynamic}.

It would be interesting to generalise the translation to Generic FJ (GFJ; FJ with Java-style generics), also considered by \textcite{featherweight-java} following \textcite{generic-java}. While generic classes seem straightforward, generic methods lead to types with higher-order polymorphism, which would hamper type inference. This displays the limitation of \emph{anonymising types}: nominal typing makes certain type system features easier than in the structurally typed case.

\section{Summary}

I have identified \textbf{duck typing} as a well-behaved pattern in dynamically typed programs, and \textbf{structural subtyping} as its appropriate model suitable for constructing expressive static type systems. I motivate this choice by presenting translations and embeddings for a pair of calculi: the nominally typed Featherweight Java, and the structurally typed Featherweight Lua. These formally reinforce intuitive beliefs about the static-dynamic and nominal-structural axes of type systems, and serve as groundwork for the thesis.

Though this dissertation does not explore the application of structural subtyping to statically typing specific dynamic languages (e.g.\@ as part of a gradual type system \cite{gradual-typing-for-objects}), it makes contributions towards it, and motivates the application of structural subtyping for this purpose. In the next chapter, I develop a type inference framework for languages featuring structural subtyping -- including Featherweight Lua.

% It has by now become evident that the aim of this thesis is not to implement a new system for typing dynamic languages directly. Instead, I identify and develop general approaches that show potential to be used for this purpose. Furthermore, they can inform the design of (possibly domain-specific) languages that can admit more flexible type systems.
  \newcommand{\mlsub}{\textsc{MLsub}}
\newcommand{\simplesub}{\textsc{Simple-sub}}
\newcommand{\mlstruct}{\textsc{MLstruct}}
\newenvironment{example}{%
\begin{tcolorbox}[%
    colback=blue!5!white,% 
    colframe=blue!60!black,%
    title=\textsc{Example}%
]%
}{%
\end{tcolorbox}%
}

\chapter{Constraint-Based Algebraic Subtyping}
\label{algebraic-subtyping}

Part of the convenience of programming with dynamic typing is the absence of type annotations. 
This comfort comes at a price -- we cannot ensure any safety guarantees at compile-time. 
The aim of this thesis is to statically type languages with similar flexibility in mind. As such, we need to recover static types in the absence of annotations -- to have \emph{implicit typing} \cite{remy-record-inference}). This is the mission statement of \textbf{type inference} \cite{tapl}.

In this chapter, I explore the problem of type inference for languages with structural subtyping. In doing so, I follow Dolan's seminal thesis on \textbf{algebraic subtyping} -- a type inference technique in the presence of subtyping and (bounded) parametric polymorphism. Specifically, I contribute the following:
\begin{itemize}
    \item A \textbf{constraint-based type inference framework} -- \inference{} -- based on the current state-of-the-art in algebraic subtyping. The framework does not fix specific type or expression languages -- it sets out some requirements for them (the \textit{signature}) and operates on an intermediate \emph{constraint language}. The constraint-based description is simple and direct, though formal and close to the implementation. As such, it is a step towards understanding when we can apply algebraic subtyping in practice.
    \item An extension of algebraic subtyping that supplements the type language with applications of \textbf{type lattice homomorphisms}, described for the framework. Using this extension, I give a method to statically type extensible records using algebraic subtyping -- a novel alternative to row polymorphism. 
\end{itemize}

We begin with background on type inference, including our constraint-based setting and a review of the developments in algebraic subtyping (Section \ref{sec:ch3background}).The rest of the chapter's contents are the description of the framework. Firstly, in Section \ref{sec:signature} I give a description of the \textbf{signature} of the source language -- the requirements on the type language. Afterwards, in Section \ref{sec:constraints} I explain the \textbf{constraint solving} approach in \inference{}, and show the homomorphism extension in Section \ref{sec:morphisms}. Finally, I state and conjecture the correctness theorems of the framework in Section \ref{sec:correctness}. 

\begin{example}
    To ease understanding, the technical text of this chapter will be interleaved with boxes like this one, containing examples using Featherweight Lua (with some described extensions).
\end{example}

\section{Background}
\label{sec:ch3background}

Since this chapter concerns describing algebraic subtyping in the framework of constraint-based type inference, I explain these two concepts. We also set up the setting of the type inference problem we consider. 

\subsection{Type inference}
\emph{Type inference} (also called \emph{type reconstruction}), at its core, concerns determining the type $\tau$ of a given expression $e$ under an environment $\Gamma$, under some typing judgement $\Gamma \vdash e : \tau$.

\paragraph{Polymorphism \& type schemes} Type inference is straightforward in a \emph{simply-typed} setting. This is often unsatisfactory -- for instance, $\mathrm{id} =  \lambda x \ldotp x$ has type $\tau \to \tau$ for any type $\tau$.
This leads us to \textbf{parametric polymorphism}, where types contain type variables (denoted $\alpha, \beta, \gamma, \dots$), which can stand for any type. 
In this setting, expressions are not only given a type $\tau$, but a \textbf{type scheme} $\sch$ which can be \textbf{instantiated} to a type $\tau$ (written $\sch \models \tau$). A type scheme stands for any of these instantiations, and we define a judgement $\Gamma \vdash e : \sigma$ for ascribing a type scheme:
$$ \dfrac{\forall \tau \ldotp\; \sch \models \tau \implies \Gamma \vdash e : \tau}{\Gamma \vdash e : \sigma} $$
As expected, $\cdot \vdash \mathrm{id} : \forall \alpha \ldotp \alpha \to \alpha$.

The classical solution to type inference under parametric polymorphism is through the Hindley-Milner (HM) type system, which relies on unifications to compute most-general type substitutions \cite{essence-of-ml-type-inference, tapl}. It underlies type systems of languages in the ML family.

In this chapter, we will consider type inference in the presence of $F_\sub$-style \textbf{bounded} parametric polymorphism (in essence, \emph{bounded quantification} of \textcite{bounded-quantification}). Our type schemes $\sch$ have a form reminiscent of Java-style generics \cite{generic-java, simple-sub}:
$$ \sch ::= \forall \overline{\tau \sub \alpha \sub \tau} \ldotp \tau $$
where the body $\tau$ of $\sch$ must not contain unquantified type variables, and unspecified lower/upper bounds are presumed $\bot$/$\top$. 
Instantiation $\models$ derives from a type assignment $\psi ::= \cdot \mid \tau/\alpha$, which satisfies the bounds for each free type variable $\alpha$. Writing $[\psi]\tau$ for a substitution in $\tau$ under $\psi$, we have:
$$ \dfrac{\overline{[\psi]\tau_\alpha^+ \sub \substx{\psi} \alpha \sub [\psi]\tau_\alpha^-}}{\forall \overline{\tau_\alpha^+ \sub \alpha \sub \tau_\alpha^-} \ldotp \tau \models \substx{\psi}{\tau}} $$
In particular, if $\sigma \models \tau$, then $\tau$ has no free type variables.

\begin{example}
    We extend FL with bounded parametric polymorphism from here onwards. 

    Consider $e = \fllam x\, \flproj{x}{\mathrm{foo}}$. Then we have both 
    $$ \cdot \vdash e : \forall \alpha, (\beta \sub \flrec{\mathrm{foo} : \alpha}) \ldotp \beta \to \alpha  \quad \text{and} \quad \cdot \vdash e : \underbrace{\forall  \alpha \ldotp \flrec{\mathrm{foo} : \alpha} \to \alpha}_\sigma $$
    And we have that $\sigma \models \flrec{\mathrm{foo} : \sint} \to \sint$ at $\sint/\alpha$.
    
    The existence of many valid type schemes points us to type scheme simplification -- here, we obtain the second type scheme by \emph{inlining} the bound on $\beta$ in the first. This is an important topic in type inference with subtyping, and one we consider at the end of Section \ref{subsec:simplification}.
\end{example}

We shall allow bounds $\tau_\alpha^+$/$\tau_\alpha^-$ in type schemes to refer to other type variables, which naturally leads us to including recursive types in the type language.
Specifically, we will consider type systems with \textbf{equirecursive types}, meaning they are infinite terms in the type language -- as opposed to isorecursive types, where recursive types are given by finite types which are (un)folded explicitly \cite{tapl}. We will write $\rec \alpha \tau$ for a type  such that instances of $\alpha$ are equal to the entire type.
\begin{example}
    We further extend FL with equirecursive $\mu$ types. 
    The type $\tau = \rec \alpha \top \to \alpha$ corresponds to the infinite type $\top \to (\top \to (\top \to \cdots)))$. Given $\cdot : e : \tau$ and $\cdot : e' : \top$ we have $\cdot : e\,e' : \tau$. Hence, $\tau$ is the type of a function that takes an infinite number of arguments.
\end{example}

We define two more concepts useful for dealing with type schemes: \emph{subsumption} and \emph{principality} (minimality). We define that $\sigma'$ \textbf{subsumes} $\sigma$, written $\sigma \subsume \sigma'$, as follows:
$$ \sigma \subsume \sigma' \iff \forall \tau \ldotp (\sigma \models \tau \impliedby \sigma' \models \tau) $$
meaning that $\sigma'$ admits all the types that $\sigma$ does.
Based on subsumption, we define the \textbf{principal} type scheme $\sigma$ for an expression $e$ as the one that subsumes all its other type schemes $\sigma'$ (it is minimal\footnote{When we take the preorder $\subsume$ on all type schemes $\sigma$ for which $\Gamma \vdash e : \sigma$.}), i.e.\@:
$$ \sigma\text{ principal} \iff \forall \sigma' \ldotp \left( \Gamma \vdash e : \sigma' \implies \sigma \subsume \sigma' \right) $$
Subsumption can be seen as a generalisation of subtyping to type schemes.
\begin{example}
Consider the following type schemes:
$$    \sigma^+ = \forall \alpha, \beta \ldotp \alpha \to \beta
\quad \sigma' = \forall \beta \ldotp \flrec{} \to \beta 
\quad \sigma'' = \forall \alpha \ldotp \alpha \to \flrec{} 
\quad \sigma^- = \flrec{} \to \flrec{} $$
Then $\sigma^+ \subsume \sigma' \subsume \sigma^-$ and $\sigma^+ \subsume \sigma'' \subsume \sigma^-$, but neither $\sigma' \subsume \sigma''$ nor $\sigma'' \subsume \sigma'$. 
\end{example}

\paragraph{Constraint-based approach} While type schemes let us describe the \emph{result} of type inference, we can use \textbf{constraints} to describe type inference \emph{problems}. To this end, we follow the approach outlined in \citetitle{essence-of-ml-type-inference} by \textcite{essence-of-ml-type-inference} (\textcite[Chapter~10]{adv-tapl}).

This chapter will focus on constraint solving in the presence of subtyping, so we give an adequately simple constraint language in Figure \ref{fig:constraints} with only one predicate -- subtyping $\tau \sub \tau$, where types $\tau$ may contain type variables. We also feature constraint conjunction $c \und c$ and allow introducing existential variables $\exists \alpha \ldotp c$. The constraint satisfaction judgement $\psi \vdash c$ (where $\psi$ binds all free type variables in $c$) defined in Figure \ref{fig:satisfaction} gives a semantics to this syntax, specifying what variable assignments $\psi$ satisfy a given constraint $c$. In constraint solving nomenclature, we are dealing with a cylindric constraint system \cite{constraint-based-hm}. 

\begin{figure}
    \centering
    \begin{align*}
        \graintro c 
        \tru
        & \text{(always-true)}
        \graitem
        \fals
        & \text{(always-false)}
        \graitem
        \tau \sub \tau 
        & \text{(subtyping)}
        \graitem
        c \und c
         & \text{(conjunction)}
        \graitem 
        \exists \alpha \ldotp c
        & \text{(existential)}
    \end{align*}
    \caption{Syntax of constraints $c$ used in this chapter.}
    \label{fig:constraints}
\end{figure}

\begin{figure}
    \centering
    $$
    \irule{CTrue}{}{\psi \vdash \tru}
    \quad
    \irule{CSub}{\substx \psi \tau \sub \substx \psi \tau'}{\psi \vdash \tau \sub \tau'}
    \quad
    \irule{CAnd}{\psi \vdash c \quad \psi \vdash c'}{\psi \vdash c \und c'}
    \quad 
    \irule{CExist}{\psi, \tau/\alpha \vdash c}{\psi \vdash \exists \alpha \ldotp c}
    $$
    \caption{Constraint satisfaction judgement $\psi \vdash c$, defined for a type variable assignment $\psi$ and constraint $c$. Note that $\psi \vdash \fals$ is false for any $\psi$ ($\fals$ signals failure of inference -- type errors), and $\psi \vdash \tru$ is true for any $\psi$.}
    \label{fig:satisfaction}
\end{figure}
 
To construct a type inference problem as a constraint $c$ from an expression $e$ in the source language and its expected type $\tau$, we use \textbf{constraint generation} $\denot{e : \tau} = c$. Crucially, we require it agrees with typing:\footnote{Some standard presentations instead modify the type-scheme judgement to involve constraints, i.e.\@ $c; \Gamma \vdash e : \sigma$ \cite{essence-of-ml-type-inference}. For simplicity of the presentation, we take specific instantiations $\substx \psi \tau$, closer to e.g.\@ \cite[Section 3.4]{constraint-based-freeze-ml}.}
$$ \psi \vdash \denot{e : \tau} \iff \substx{\psi}\Gamma \vdash e : \substx \psi \tau $$

Since in type inference we do not know the specific type $\tau$, we can introduce it as a free variable in a constraint $\denot{e : \alpha}$. Analogously, if an expression contains `type holes' (like unannotated types of function parameters) these can be filled with type variables and constrained appropriately \cite{tapl}.

\begin{example}
    Consider $e = \fllam{x: \beta} \flproj{x}{\mathrm{quack}}\,\{\}$. Then we might have:
    $$ \denot{e : \alpha} = \exists \gamma \ldotp \beta \sub \flrec{\mathrm{quack}: \gamma} \und \exists \delta \ldotp \gamma \sub \flrec{} \to \delta \und \delta \sub \alpha $$
    where complete constraint generation for FL is defined in Appendix \ref{extra:fl-constraints}. Note that subtyping constraints and introduced type variables roughly follow the program dataflow \cite{mlsub}.

    We will later see this constraint system can be rewritten to $\exists \gamma \ldotp \flrec{\mathrm{quack} : \flrec{} \to \gamma} \to \gamma \sub \alpha$, and hence $\cdot \vdash e : \forall \gamma \ldotp \flrec{\mathrm{quack} : \flrec{} \to \gamma} \to \gamma$.
\end{example}\todo[color=green]{extra}

Lastly, we shall have \emph{constraint equivalence} $\cstreq$ such that:
$$ c \cstreq c' \iff  \forall \psi \ldotp \; (\psi \vdash c \iff \psi \vdash c') $$

The description of \inference{} is given so that we could instantiate an existing constraint-based type inference framework like $\mathrm{HM}(X)$ of \textcite{constraint-based-hm}. Thus, we do not consider let-polymorphism (like \textcite{dolan-thesis} and \textcite{simple-sub} do algebraic subtyping) which $\mathrm{HM}(X)$ could yield \enquote{for free}.

% It is worth noting here that this treatment of type inference is a bit different to the common Hindley-Milner type inference. There, we reason about unification and most general substitutions, but are also limited to an equality (rather than subtyping) predicate in constraints. Dolan's \emph{algebraic subtyping} (which we expand on in the next section) also reasoned about \emph{biunification} \emph{bisubstitutions}, more recent work moved towards a constraint-based approach, and this is the one we shall follow.

\subsection{Algebraic subtyping}

Combining bounded parametric polymorphism with both principal type inference and decidability of type scheme subsumption proved to be a difficult problem, which led to a general distrust in implicit subtyping as part of language design \cite{mlstruct} -- so much so that research would avoid subtyping due to its problematic interaction with type inference (see e.g.\@ \cite[Section~3.5]{linear-haskell}). Seminal work in the area is by \textcite{pottier-thesis}, who set out a framework for type inference under subtyping, but did not reach a satisfactory solution. The problem was ultimately resolved by \textcite{dolan-thesis} in his thesis.

\subsubsection{Original work (Dolan)}
There are two (closely tied) core principles guiding Dolan's approach \cite[Section~1.3]{dolan-thesis}: \begin{description}
    \item[Extensibility] Dolan identified that a core problem in previous solutions was closed-world reasoning on the language of types \cite[Section~1.3.1]{dolan-thesis}. Thus, he requires \emph{extensibility}: that for considered type systems, extending their type language preserves typing of programs.
    \item[Algebra before syntax] Furthermore, Dolan found that the previous focus on syntactic approaches to defining the type language and subtyping over it neglected ensuring these are well-behaved \cite[Section~1.3.2]{dolan-thesis}. Thus, he argued for an algebraic approach to constructing the type language. Indeed, extensibility motivated many of these algebraic properties \cite[Section~2.1.5]{dolan-thesis}.
\end{description}
Using these principles and a careful formal treatment relying on abstract algebra, Dolan invented \mlsub{} -- a statically typed language with support structural subtyping, building on core ML, and boasting ML-style type inference with bounded parametric polymorphism and decidable type scheme subsumption.

Two assumptions underlie \mlsub{} which enable its properties: subtyping forming a \textbf{distributive lattice}, and the \textbf{polarity restriction} of the type language. While Dolan uses an HM-like presentation to type inference (with \emph{bi}unification and \emph{bi}subsitution), I illustrate these assumptions in a constraint-based setting. \begin{description}
    \item[Distributive lattice] While constructing a lattice of types for conveniently well-behaved subtyping relations was standard, Dolan found it crucial for the lattice to also be \emph{distributive} \cite[Section~3.2]{dolan-thesis}.\footnote{Looking ahead, \ref{fig:boolean-laws} lists the laws of a distributive lattice ($\mathsf L \cup \mathsf B$).} While his main motivation is to ensure subsumption is decidable, this is a useful assumption in general.
    \item[Polarity restriction] Furthermore, Dolan uses a \emph{polar types} construction due to \textcite{pottier-thesis} \cite[Section~5.1]{dolan-thesis}. It means that we split the type language into positive $\tau^+$ and negative $\tau^-$ types -- corresponding to \emph{outputs} (for which we wish to give lower-bound on the type) and \emph{inputs} (dually: upper-bound). We then restrict joins and meets so that $\tau^+ ::= \cdots \mid \tau^+ \join \tau^+$ and $\tau^- ::= \cdots \mid \tau^- \meet \tau^-$,
    and also restrict type constructors so that polarities agree with their variance.\footnote{Covariance preserves polarity, but contravariance flips it: for functions, $\tau^+ ::= \cdots \mid \tau^- \to \tau^+$ and $\tau^- ::= \cdots \mid \tau^+ \to \tau^-$}
    
    The polarity restriction ensures subtyping constraints have form $\tau^+ \sub \tau^-$ (cf.\@ output-to-input dataflow \cite[Section~1.1]{dolan-thesis}). Hence, we can \emph{split} meets and joins via lattice laws \cite{simple-sub}: 
    \begin{align*}
        \tau' \join \tau'' \sub \tau \iff \tau' \sub \tau \text{ and } \tau'' \sub \tau \qquad
        \tau \sub \tau' \join \tau'' \iff \tau \sub \tau' \text{ and } \tau \sub \tau''
    \end{align*}

    A practical consequence of the polarity restriction is that we cannot type functions as $\alpha \to \alpha \meet \beta$ -- i.e.\@ ones \enquote{strengthening} the type of a value. The assumption was thus too restrictive for my approach to typing extensible records, and I chose to look for follow-up work. 
\end{description}

% \color{red}
% \subsubsection{Dolan's original work}

% The key aspects of Dolan's approach were to restrict the subtyping order to form a \textbf{distributive lattice} algebra and to impose the \textbf{polarity restriction} on types. Using these, \textcite{mlsub} gave a language design -- \mlsub{} -- with all the desired properties.

% \paragraph{Distributive lattice} Like Dolan, we also consider a distributive lattice of types (with some extensions). This means that our type language contains meets $\meet$ (least upper bounds) and joins $\join$ (greatest lower bounds):
% $$ \tau ::= \cdots \mid \tau \meet \tau \mid \tau \join \tau $$
% where we consider types under an equivalence relation given by algebraic laws of the distributive lattice. 

\begin{example}
    In the distributive lattice of FL types (see Appendix \ref{extra:fl-constraints}), we compute:
    \begin{align*}
        (\top \to \bot) \meet (\alpha \to \beta) &= \top \to \bot \\ 
        \flext{\mathrm{quack} : \top, \mathrm{walk} : \flrec{}}{\flabsent} \meet \flext{\mathrm{quack} : \top \to \top}{\flftop} &= \flext{\mathrm{quack}: \top \to \top, \mathrm{walk}: \flrec{}}{\flabsent} \\ 
        \flrec{\mathrm{foo}: \flrec{}} \join (\top \to \top) &= \top
    \end{align*}
\end{example}\todo[color=green]{extra}

% \paragraph{Polarity restriction} The polarity restriction (used by both \textcite{dolan-thesis} and \textcite{pottier-thesis}) splits types $\tau$ into positive types $\tau^+$ and negative types $\tau^-$, so that we only consider subtyping constraints of form $\tau^+ \sub \tau^-$. Intuitively, positive types are used for \emph{outputs} (thus, we use them as a lower bound, e.g.\@ a function's result), while negative types are used for \emph{inputs} (upper bounds, e.g.\@ function's argument).

% Crucially, \textcite{mlsub} not only adapt \mlsub{}'s type constructors to respect these polarities (in a manner matching their covariance/contravariance; e.g.\@ $\tau^+ ::= \cdots \mid \tau^- \to \tau^+$) -- this way, we only ever have subtyping constraints with matched polarities, i.e.\@ $\tau^+ \sub \tau^-$. Furthermore, they only permit joins in positive types, and meets in negative types:
% $$ \tau^+ ::= \cdots \mid \tau^+ \join \tau^+ \quad \tau^- ::= \cdots \mid \tau^- \meet \tau^- $$
% Together with the properties of the lattice algebra, this allows us to cleanly decompose and solve constraints $\tau^+ \sub \tau^-$ via the following pair of properties:
% \begin{align*}
% \tau' \join \tau'' \sub \tau \;&\iff \tau' \sub \tau \und \tau'' \sub \tau \\
% \tau \sub \tau' \meet \tau'' \;&\iff \tau \sub \tau' \und \tau \sub \tau'' 
% \end{align*}
% This property of lattices remains crucial in approaches that follow algebraic subtyping.

% However, the restriction of where meets and joins can occur in types -- and thus constraint solving -- means the language and its type system has to be carefully designed and certain types might not be expressible. 
% Notionally, to type-check updating the field of a record, we will need a primitive of a form that further constrains the type to a subtype with the field updated:
% $$ \forall \alpha, \beta \ldotp \alpha \to \alpha \land \flrec{\mathrm{foo}: \beta}  $$
% Such a type breaks the polarity restriction, since a negative type (with a $\meet$) occurs in a positive position (since the inferred type of an expression is a lower bound).
% Note this is not the type we actually use for typing extensible records, as there are problems with it -- we expand on this in Section \ref{sec:morphisms}.

% \paragraph{Formal treatment} Dolan's techniques revolved around the use of abstract algebra to prove the correctness of his techniques. For type inference itself, he parted with Pottier's constraint-oriented methods, and instead used biunification and bisubstitutions -- generalisation of standard notions in Hindley-Milner type inference to a setting with subtyping \cite{tapl, dolan-thesis}. 

\subsubsection{Later work (Parreaux, Chau)}

% \subsubsection{Later work}
% The name \textbf{algebraic subtyping} refers to these restrictions and techniques that Dolan invented, but this report also uses the name for closely related methods which followed, particularly due to \textcite{simple-sub} (\simplesub{}) and \textcite{mlstruct} (\mlstruct{}). These methods are connected, particularly by assumptions about the subtyping order forming a well-behaved algebra of types. We now explain the significance of their work as expanding on Dolan.

% \paragraph{\simplesub{}} 
% Dolan's approach, although formally solid, leads to a complex implementation. This is shown by \textcite{simple-sub}, when they find demonstrable bugs in Dolan's reference implementation, and introduce a comparatively simpler approach operating on a \emph{constraint graph}. 

While Dolan's work is foundational to this line of work, the approach in \inference{} is more closely descended from follow-up work: \simplesub{} of \textcite{simple-sub}, and \mlstruct{} of \textcite{mlstruct}. I highlight the following developments and their relation to my work: \begin{description}
    \item[Constraint graphs] \textcite{simple-sub} found that the \emph{biunification} approach Dolan uses can be difficult to implement and extend. He proposes an alternative approach in \simplesub{} that relies on explicitly introducing subtyping constraints and imperatively updating a \emph{constraint graph}. I also do not use Dolan's biunification (due to the next point), but my method does not rely on mutation.\footnote{While it is arguably a stylistic choice, purity seems to make the implementation easier to reason about.}
    \item[Boolean algebra] In Parreaux's constraint graph setting, \textcite{mlstruct} propose \mlstruct{}, featuring \emph{complement (negation) types} and removal of the polarity restriction. They use \simplesub{}-like constraint solving by requiring complements (with meets and joins) to form a \emph{Boolean algebra}\footnote{A bounded, distributive and complemented lattice.}. My approach to constraint solving directly inherits from theirs. However, I attempt to clarify that negations are only added to the type language as a \emph{free extension} of the distributive lattice of type constructors, and explain why they are a safe addition (Section~\ref{subsec:oh-god-complements}).
    \item[Non-extensibility] For Dolan, extensibility also means that \enquote{useless} types like $(\alpha \to \beta) \join \flrec{\ell : \gamma}$ should not be \emph{equal} to $\top$ \cite[Section~1.4.1]{dolan-thesis} -- even though they can only be eliminated as such. \textcite{mlstruct} disagree with this consequence (but not others, e.g.\@ distributivity), arguing that provides better user experience and eases constraint solving. I also simplify such types, but do not share their scepticism of extensibility \cite{simple-sub}, agreeing with Dolan that it is an important principle. 
\end{description}

\section{Signature}
\label{sec:signature}

We consider the external side of the \inference{} framework -- the \emph{signature}, i.e.\@ the requirements it entails on the source language. 
All constructs described here are available as \textbf{data} upholding certain \textbf{laws} when we later describe constraints and how we solve them in Section~\ref{sec:constraints}. To this end, I mainly endeavour to generalise the properties that are necessary for the \mlstruct{} solver to work.

\subsection{Type constructors}

Firstly, following standard practice \cite{essence-of-ml-type-inference, constraint-based-freeze-ml}, we will abstract away \emph{type constructors} in the type language. We will denote them $\constr[\overline \tau]$, where $\overline \tau$ stand for the list of types that occur within.
\begin{example}
    In FL we have the function and record type constructors. Writing \enquote{$\cdot$} for \emph{type holes} in constructors:
    $$
       \constr ::= \top \mid \bot \mid {\cdot} \to {\cdot} \mid \flext{\ell : \dot \phi}{\dot \phi} \qquad \dot \phi ::= \flftop \mid \flfbot \mid \cdot \mid \flabsent
    $$
    Writing $\constr[\overline \tau]$, we plug in $\overline \tau$ into each \enquote{$\cdot$} in $\constr$ in order. This establishes that type constructors $\constr$ non-opaquely contain subterms $\tau$. Here are example types $\constr[\overline \tau]$:
    $$
        (\cdot \to \cdot)[\top, \bot] = \top \to \bot \qquad \flext{\mathrm{foo}: \cdot, \mathrm{bar} : \flabsent}{\flftop}[\top \to \top] = \flext{\mathrm{foo}: \top \to \top, \mathrm{bar} : \flabsent}{\flftop}
    $$
\end{example}

We will require that type constructors $\constr[\overline \tau]$ form a distributive lattice $(\top, \bot, \cjoin, \cmeet)$, where $\top$ and $\bot$ are nullary type constructors and $\cjoin$ and $\cmeet$ are closed binary operators on type constructors. Furthermore, we require a \textbf{decomposition} operator $\constr[\overline \tau] \cdecomp \constr[\overline \tau] = c$, which decomposes a subtyping constraint between two type constructors into an equivalent one:
$$ (\constr'\left[\overline {\tau'}\right] \sub \constr''\left[\overline {\tau''}\right]) \cstreq (\constr\left[\overline {\tau'}\right] \cdecomp \constr'\left[\overline {\tau''}\right]) $$
where the resulting constraints contain structurally smaller type constructors.\footnote{We do not formalise this property, but it would be necessary to do so to prove the constraint solving process terminates.} Decomposition is where type errors may be raised in the system, e.g.\@ in the case $\top \cdecomp \bot = \fals$. Together with the type constructor lattice and constraint decomposition we will be able to effectively massage constraints involving type constructors. 
\begin{example}
    The type constructor lattice given by $\cmeet$ and $\cjoin$ agrees with $\meet$ and $\join$ on types, e.g.:
    \begin{align*}
       (\top \to \top) \meet (\bot \to \bot) 
       &\typeq (\cdot \to \cdot)[\top, \top] \cmeet (\cdot \to \cdot)[\bot, \bot] = (\cdot \to \cdot)[\top \cjoin \bot, \top \cmeet \bot] \\
       &= (\cdot \to \cdot)[\top, \bot] \typeq \top \to \bot 
    \end{align*}
    Constraints on function type constructors decompose as such:
    $$ (\tau \to \pi) \cdecomp (\tau' \to \pi') = \tau' \sub \tau \und \pi \sub \pi' $$
\end{example}

\begin{figure}
    \centering
    \begin{align*}
    \graintro \tau 
             \alpha & \text{(variable)}
    \graitem \constr[\overline \tau] & \text{(constructor)}
    \graitem \tau \join \tau & \text{(join)}
    \graitem \tau \meet \tau & \text{(meet)}
    \graitem \lnot \tau & \text{(complement)}
    \end{align*}
    \caption{Syntax of types $\tau$ in \inference{}.}
    \label{fig:signature-types}
\end{figure}

\begin{figure}
    \centering
    % $$ \mathbb B = (\tau, \sub, \top, \bot, \join, \meet) $$ 
    $$ \boxed{\tau \typeq \tau} $$
    $$ \renewcommand\arraystretch{1.1} \begin{array}{cr}
    \tau \join (\tau' \join \tau'') \typeq (\tau \join \tau') \join \tau'' \quad 
    & \text{($\mathsf L$: associativity \mbox{$\join$})}
    \\
    \tau \meet (\tau' \meet \tau'') \typeq (\tau \meet \tau') \meet \tau''
    \tau \join \tau' \typeq \tau' \join \tau  \quad 
    & \text{($\mathsf L$: associativity \mbox{$\meet$})} 
    \\
    \quad
    \tau \meet \tau' \typeq \tau' \meet \tau 
    & \text{($\mathsf L$: commutativity)}
    \\
    \tau \join (\tau \meet \tau') = \tau
    \quad 
    \tau \meet (\tau \join \tau') = \tau
    & \text{($\mathsf L$: absorption)}
    \\ 
    \tau \join \bot \typeq \tau
    \quad 
    \tau \meet \top \typeq \tau 
    & \text{($\mathsf B$: bounds)} 
    \\
    \tau \meet (\tau' \join \tau'') \typeq (\tau \meet \tau') \join (\tau \meet \tau'')
    & \text{($\mathsf D$: distributivity)}
    \\
    \tau \join \comp \tau = \top 
    \quad
    \tau \meet \comp \tau = \bot
    & \text{($\mathsf C$ complements)} 
    \\ 
    \constr_1\left[\overline {\tau'}\right] \join \constr_2\left[\overline {\tau''}\right] \typeq \constr_1\left[\overline {\tau'}\right] \cjoin \constr_2\left[\overline {\tau''}\right]
    & \text{(type constructor $\cjoin$/$\join$)}
    \\
    \constr_1\left[\overline {\tau'}\right] \meet \constr_2\left[\overline {\tau''}\right] \typeq \constr_1\left[\overline {\tau'}\right] \cmeet \constr_2\left[\overline {\tau''}\right]
    & \text{(type constructor $\cmeet$/$\meet$)}
    \\[0.5em] 
    \dfrac{\tau \typeq \tau'}{E[\tau] \typeq E[\tau']} 
    & \text{(congruence)}
    \end{array} $$
    $$ \equivctx ::= \ctxhole \mid \constr[\overline \tau, \ctxhole, \overline \tau] \mid \equivctx \join \tau \mid \tau \join \equivctx \mid \equivctx \meet \tau \mid \tau \meet \equivctx \mid \comp \equivctx $$
    \caption{Laws of the Boolean algebra of types and the lattice of type constructors, forming the equivalence $\tau \typeq \tau$. Equivalence contexts $\equivctx$ are used to specify the congruence rule. We have laws of a lattice ($\mathsf L$) that is bounded ($\mathsf B$), distributive ($\mathsf D$), and complemented ($\mathsf C$) -- altogether, a Boolean algebra.}
    \label{fig:boolean-laws}
\end{figure}

\subsection{Type language}

With the necessary structure of type constructors, we describe the syntax of types itself (Figure \ref{fig:signature-types}). Following \textcite{mlstruct}, we define it so that it forms the free Boolean algebra over type constructors and variables\footnote{Like \textcite{dolan-thesis}, our variables are \emph{opaque} in the lattice -- we do not make a closed-world assumption about their possible values.}. This algebra satisfies Boolean algebra laws and the laws of the type constructor lattice (Figure \ref{fig:boolean-laws}) under type equivalence $\tau \equiv \tau$.\footnote{Following this, the algebra of types is free extension of the algebra of type constructors with variables and complements.} As a coherence condition, subtyping $\sub$ must agree with the type algebra and the typing judgement:
$$ \begin{array}{c}
   \tau \sub \pi \iff \tau \typeq \tau \meet \pi \iff \pi \typeq \tau \join \pi \\[5pt]
   \dfrac{\Gamma \vdash e : \tau \quad \tau \sub \tau'}{\Gamma \vdash e : \tau'}
\end{array} $$

\subsection{Inclusion of complement types} 
\label{subsec:oh-god-complements}
Complement (or negation) types are a somewhat controversial addition in a type system. However, following \textcite{mlstruct}, our complement types have algebraic foundation, as opposed to a set-theoretic one. The two have different interpretations of subtyping:\footnote{The algebraic interpretation follows by: $\pi \sub \comp \tau \implies \pi \meet \tau \sub \tau \meet \comp \tau \implies \pi \meet \tau \sub \bot$, and $\pi \meet \tau \sub \bot \implies (\pi \meet \tau) \join \comp \tau \sub \comp \tau \implies (\pi \join \comp \tau) \meet (\pi \join \comp \tau) \sub \comp \tau \implies (\pi \join \comp \tau) \meet \top \sub \comp \tau \implies \pi \join \comp \tau \sub \comp \tau \implies \pi \sub \comp \tau$.}
\begin{align*}
    \pi \sub \dot \comp \tau \;&\iff \text{not } \pi \sub \tau & \text{(set-theoretic)} \\
    \pi \sub \comp \tau \;&\iff \pi \meet \tau \sub \bot & \text{(algebraic)} 
\end{align*}
Set-theoretic-like complement types notoriously make constraint solving harder, smuggling negations into the underlying logic of constraint satisfaction. On the other hand, our algebraic complement types provide a weaker condition, but will be of great help to constraint solving -- as we explore in Section \ref{fig:constraints}.

\begin{example}
    Take $\pi = \flext{\mathrm{foo}: \flabsent }{\flftop}$ and $\tau = \flext{\mathrm{foo}: \top}{\flftop}$. The two interpretations of $\pi \sub \comp \tau$ disagree: \begin{description}
        \item[Set-theoretic] It is not the case that $\pi \sub \tau$, thus we have $\pi \sub \dot \comp \tau$. $\checkmark$
        \item[Algebraic] We have $\pi \meet \tau = \flext{\mathrm{foo} : \flfbot }{\flftop} \not\sub \bot$, so we do not have $\pi \sub \comp \tau$. $\lightning$
    \end{description}
    Note that $\comp$ admits fewer subtypes than $\dot \comp$, since for any sets $A$ and $B$: 
    $$A \cap B = \varnothing \implies \lnot(A \subseteq B) \qquad\text{thus}\qquad \tau \sub \comp \pi \implies \tau \sub \dot\comp \pi $$ 
    % Intuitively, algebraically we require an empty (bottom) intersection between the types, not non-subset.
\end{example}

\subsection{Summary} To summarise the requirements of \inference{}'s signature: \begin{itemize}
    \item Type constructors must form a distributive lattice.
    \item We use a type language that forms a Boolean algebra.
    \item We must have an implicit subtyping rule in the type system.
\end{itemize} 
To determine these, I was inspired by the requirements of techniques of \textcite{mlsub} and \textcite{mlstruct}.
Later, we see that both Fabric (Chapter \ref{fabric}) and Star (Chapter \ref{star}) successfully implement the signature of \inference{}, showing it is reasonable in practice.

\section{Constraint solving}
\label{sec:constraints}

We now describe the \mlstruct{}-inspired constraint solving process used in \inference{} using the already described setting of the constraint language and signature. 

I largely follow the tradition of the framework by \textcite{pottier-framework} and the Boolean algebraic techniques of \textcite{mlstruct} in \mlstruct{}. The key difference to prior art is the use term rewriting of constraints, leading to a simpler presentation. Furthermore, in Section \ref{sec:morphisms} I give the main contribution of the framework: extension of the type language and constraint solving with homomorphism applications. Giving this extension is made much easier by a constraint-based presentation.

The section is structured as follows: \begin{description}
    \item[\ref{subsec:rewriting} Massaging] We first show that all subtyping constraints $\tau \sub \tau$ can be reduced to a conjunction of \emph{variable-bound} constraints (of syntax $\tau \sub \alpha \,\mid\, \alpha \sub \tau$ -- as in bounded parametric type schemes).
    \item[\ref{subsec:normalisation} Plumbing] Then, we show how we manipulate and combine these variable-bound constraints into a list-of-bounds normal form. We then perform the \emph{closure} computation.
    \item[\ref{subsec:simplification} Solutions] Lastly, we briefly consider how we can extract a type scheme from a normalised generated constraint, and how these can be simplified.
\end{description} 
Altogether, I reproduce the constraint solving of \textcite{mlstruct} in \mlstruct{} in a constraint-based style that is simpler and easier to extend. 

Constraint solving is given in terms of a small-step term rewriting relation $\step$. The solving process is given by its reflexive-transitive closure, $\step^*$ (\ref{fig:solver}).
To show constraint solving always yields correct result, we rely on a theorem that $\step^*$ preserves constraint satisfaction (with simple proofs for individual steps $\step$):\todo[color=green]{extra}
\begin{theorem}[Semantic preservation]
    If $c \step^* c'$, then $c \cstreq c'$.
\end{theorem}
We shall now consider various properties of constraint solving steps, giving proofs by construction -- these directly lead to a practical implementation.

\begin{figure}
    \centering
    $$ \boxed{c \step c} $$
$$ \renewcommand\arraystretch{1.5} \begin{array}{cr}
\dfrac{\tau_1 \equiv \tau_1' \quad \tau_2 \equiv \tau_2'}{\tau_1 \sub \tau_2 \step \tau_1' \sub \tau_2'} 
& \text{(equivalence)}
\\[4pt]
\tru \und c \step c 
& \text{(identity $\tru$ of $\undsym$)}
\\
\fals \und c \step \fals
& \text{(annihilating $\fals$ of $\undsym$)}
\\
c \und (c' \und c'') \step (c \und c') \und c'' 
& \text{(associative $\undsym$)}
\\
c \und c' \step c' \und c
& \text{(commutative $\undsym$)}
\\
c \und c \step c
& \text{(idempotent $\undsym$)}
\\
\exists \alpha \ldotp \tru \step \tru 
\quad
\exists \alpha \ldotp \fals \step \fals
& \text{(always-exists)}
\\
(\exists \alpha \ldotp c) \und c' \step \exists \alpha \ldotp (c \und c')
& \text{(factor-out)}
\\ 
\tau \sub \tau' \und \tau' \sub \tau'' \step (\tau \sub \tau' \und \tau' \sub \tau'') \und \tau \sub \tau'' 
& \text{(transitivity)}
\\
T'\left[\overline {\tau'}\right] \sub T''\left[\overline {\tau''}\right] 
\step T'\left[\overline {\tau'}\right] \triangleleft T''\left[\overline {\tau''}\right]
& \text{(decomposition)}
\\
\comp \tau \sub \tau' \step \comp \tau' \sub \tau  \quad \comp \tau' \sub \tau \step \comp \tau \sub \tau'
& \text{(negation)}
\\
\tau \meet \tau' \sub \tau'' \step \tau \sub \comp \tau' \join \tau''
\qquad
\tau \sub \tau' \join \tau'' \step \tau \meet \comp \tau' \sub \tau''
& \text{(swapping)}
\\
\tau' \join \tau'' \sub \tau \step \tau' \sub \tau \und \tau'' \sub \tau
\qquad
\tau \sub \tau' \meet \tau'' \step \tau \sub \tau' \und \tau \sub \tau''
& \text{(splitting)} 
\\
\tau' \sub \tau \und \tau'' \sub \tau \step \tau' \join \tau'' \sub \tau
\qquad
\tau \sub \tau' \und \tau \sub \tau'' \step \tau \sub \tau' \meet \tau''
& \text{(combining)} 
\\[4pt]
\dfrac{c \step c'}{C[c] \step C[c']}
& \text{(congruence)} 
\\[4pt]
\solvectx ::= \ctxhole \mid \solvectx \und c \mid c \und \solvectx \mid \exists \alpha \ldotp \solvectx 
& \text{(solving contexts)}
\end{array} $$

    \caption{Definition of constraint solving steps $\step$. Congruence for $\step$ is given by constraint solving contexts $\solvectx$.}
    \label{fig:solver}
\end{figure}

\newcommand{\typcnf}{\tau_\text{cnf}}
\newcommand{\typcnfcls}{\tau_\text{cnf-clause}}
\newcommand{\typclsvars}{\tau_\text{cnf-clause-vars}}
\newcommand{\typvar}{\tau_\text{var}}
\newcommand{\vbnd}{V}
\newcommand{\vbnds}{\vbnd_{\undsym}}

\subsection{Massaging}
\label{subsec:rewriting}
We define the syntax of a \emph{variable bound} $\vbnd$ and \emph{variable bounds} $\vbnds$ as:
\begin{align*}
    \graintro \vbnd \tau \sub \alpha \mid \alpha \sub \tau \\
    \graintro {\vbnds} \tru \mid V \mid \vbnds \und \vbnds
\end{align*}
We shall show by construction that:
\begin{theorem}[Subtyping constraints yield variable bounds]
    Given $c = (\tau \sub \pi)$, we have that:
    $$ c \step^* \fals \;\text{ or }\; \exists c' = \vbnds \ldotp c \step^* c' $$
\end{theorem}
Let us call type constructors and type variables \emph{atoms}.
The proof sketch is as follows:
\begin{enumerate}
    \item Show all subtyping constraints are equivalent to $\top \sub \typcnf$,\footnote{Notice that $\top \sub \tau \iff \top = \tau$.} 
    for a type $\typcnf$ in \emph{conjunctive normal form} (or a \emph{meet-of-joins-of-atoms}) -- as defined in Figure \ref{fig:type-cnf}.
    \item From there, we split the constraint into a conjunction of constraints of the form $\top \sub \typcnfcls$, where $\typcnfcls$ is a CNF clause (or a \emph{join-of-atoms}).
    \item At this point, we use \emph{swapping} to give variable bounds for each occurring variable, yielding $\vbnds$.
\end{enumerate}
The technique is essentially the same as \textcite{mlstruct}, but I give a presentation that is generalised (up to the signature) and more direct (constraint-based). Just as them, I call this process \emph{constraint massaging}. 
% We now give a sketch of the main aspects of the construction.

\begin{figure}
    \centering
    \begin{align*}
    \graintro{\typcnf} \top \mid \typcnfcls \meet \typcnf \\
    \graintro{\typcnfcls} \constr \left[ \overline{\typcnf} \right] \join { \comp \constr \left[ \overline{\typcnf} \right]} \join \typclsvars \\
    \graintro{\typclsvars} \bot \mid \typvar \join \typclsvars \\
    \graintro{\typvar} \alpha \mid \comp \alpha
    \end{align*}
    \caption{Grammar of the conjunctive normal forms (CNF) for types $\tau$ -- a \emph{meet-of-clauses} $\typcnf$, where clauses $\typcnfcls$ are \emph{joins-of-atoms}. Clauses only one type constructor $\constr$ at each polarity, but many variables. We additionally require that no $\typvar$ occurs twice in $\typclsvars$.}
    \label{fig:type-cnf}
\end{figure}

\begin{proof}
We shall write $\mathrm{cnf}(\tau)$ for the $\typcnf$ such that $\tau \equiv \typcnf$. CNF can be built compositionally, analogically to usual Boolean algebraic techniques.\footnote{There is an analogous \emph{disjunctive normal form} (DNF) in my implementation -- while not necessary, it can be more size-efficient.} To transform any $c = \tau \sub \pi$ into $\top \sub \typcnf$ we step like so:
$$ \tau \sub \pi \stackrel{\text{swap}}{\step} \top \sub \comp \tau \meet \pi \stackrel{\text{equiv.}}{\step} \top \sub \mathrm{cnf}(\comp \tau \meet \pi) $$
at which point we reach a $c' = \top \sub \typcnf$ equivalent to $c$. By simple induction we can see that we can split this into a conjunction of constraints $\top \sub \typcnfcls$, since:
$$ \top \sub \typcnfcls \meet \typcnf \;\stackrel{\text{split}}{\step}\; \top \sub \typcnfcls \und \top \sub \typcnf $$
It remains to show $\top \sub \typcnfcls \step^* \vbnds$. Let $\typcnf = \constr' \left[ \overline{\typcnf'} \right] \join { \comp \constr'' \left[ \overline{\typcnf''} \right]} \join \typclsvars$. By cases: 
\begin{itemize}
    \item If $\typclsvars = \bot$, then:
    $$ \top \sub \constr' \left[ \overline{\tau'_\text{cnf}} \right] \join \comp \constr'' \left[ \overline{\tau''_\text{cnf}} \right] \join \bot \stackrel{\text{equiv.}}{\step} \cdots \stackrel{\text{swap}}{\step} \cdots \stackrel{\text{decomp.}}{\step} \constr'' \left[ \overline{\tau''_\text{cnf}} \right] \cdecomp \constr' \left[ \overline{\tau'_\text{cnf}} \right] $$
    and we proceed recursively at the result of $\cdecomp$, assuming it is well-behaved so we eventually terminate. This is the case where we might reach a \enquote{type-error} constraint $\fals$.
    \item If $\typclsvars = \typvar \join \typclsvars'$, the constraint is satisfiable ($\typvar \mapsto \top$). We return a conjunction of \textbf{separate}\footnote{It is \emph{sufficient} to take one $\typvar$ -- but any choice would be nondeterministic, so to find a principal type scheme we get all of~them.} constraints for each $\typvar$. We have one of two cases: \begin{itemize}
        \item If $\typvar = \alpha$, then $\top \sub \alpha \join \tau \stackrel{\text{swap}}{\step} \comp \tau \sub \alpha$.
        \item If $\typvar = \comp \alpha$, then $\top \sub \comp \alpha \join \tau' \stackrel{\text{swap}}{\step} \alpha \sub \tau'$.
    \end{itemize}
\end{itemize}
\end{proof}

\begin{example}
Here is an example of constraint massaging. 
Starting from:
$$ c = (\alpha \meet \flrec{} \to \beta) \sub (\flrec{\mathrm{foo}: \gamma} \to \top) $$
we decompose:
$$ c \step c' = \underbrace{\flrec{\mathrm{foo}: \gamma} \sub \alpha \meet \flrec{}}_{c''} \und \beta \sub \top $$
We consider the first conjunct $c''$ (the latter is solved), and transform into $\top \sub \typcnf$:
$$ c'' \step^* c''_\text{cnf} = \top \sub \underbrace{\left( \alpha \join \comp \flrec{\mathrm{foo}: \gamma} \right)}_\text{(1)} \meet \underbrace{\left( \flrec{} \join \comp \flrec{\mathrm{foo}: \gamma} \right)}_\text{(2)} $$
Splitting into clauses (1) and (2):
$$ \begin{cases} 
    \top \sub \alpha \join \comp \flrec{\mathrm{foo}: \gamma} \step \flrec{\mathrm{foo}: \gamma} \sub \alpha
    \\
    \top \sub \flrec{} \join \comp \flrec{\mathrm{foo}: \gamma} \step \flrec{\mathrm{foo}: \gamma} \sub \flrec{} \step \flrec{\mathrm{foo}: \gamma} \cdecomp \flrec{} = \tru
\end{cases}$$
Thus:
$$ c \step^* (\flrec{\mathrm{foo}: \gamma} \sub \alpha \und \tru) \und \beta \sub \top $$
Note that while negations appeared in intermediate constraints, they are not in the output.
\end{example}

\subsection{Plumbing}
    \label{subsec:normalisation}
    
\newcommand{\sbnd}{W}

We shall now consider the normalisation of variable bounds and the transitive closure computation in the style of \textcite{pottier-framework}. The use of these techniques was first described for algebraic subtyping by \textcite{simple-sub} for \simplesub{} (and thus inherited by \mlstruct{}) -- I adapt them to the constraint-based presentation.

Lists-of-bounds $\sbnd$ have syntax:
$$ \sbnd ::= \tru \mid (\tau \sub \alpha \und \alpha \sub \tau) \und \sbnd $$
constrained such that no $\alpha$ occurs twice, and $\alpha$s are sorted under a fixed total ordering. We have only shown that for subtyping constraints reduce to (multiset-like) variable bounds $V^*$, but we can strengthen this:
\begin{lemma}[Normalisation of bounds]
For any $c = \vbnds$, there exist $c' = \sbnd$ such that $c \step^* c'$.
\end{lemma}
\begin{proof}
    Straightforward by $\undsym$ forming a commutative monoid and by the \emph{combining} step.
\end{proof}
To move towards general constraints, we also give a lemma that existentials can be lifted to the top-level:
\begin{lemma}[Top-level existentials]
    For any $c$, there exist $c'$ such that $c'$ has no existentials and $c \step^* \overline{\exists{\alpha \ldotp {}}} c'$.
\end{lemma}
\begin{proof}
    This is simplified by the constraint language only featuring conjunctions and existentials besides the subtyping predicate. Straightforward using the fact $\exists$ can always be \emph{factored out} from conjunctions.
\end{proof}
Finally, we give the normal form of constraints $\hat c$ -- a list-of-bounds with top-level existentials:
$$ \hat c ::= \mathbf F \mid \overline{\exists \alpha \ldotp {}} \sbnd $$
and we state that \textbf{constraints normalise}:
\begin{theorem}[Normalisation of constraints]
For any $c$, there exist $\hat c$ such that $c \step^* \hat c$.
\end{theorem}
\begin{proof}
    By factoring out existentials to the top-level, and normalising the variable bounds.
\end{proof}
Lastly, whenever we are in a list-of-bounds form, we can invoke transitivity. This yields an additional constraint $\tau^+_\alpha \sub \tau^-_\alpha$ for each $\tau^+_\alpha \sub \alpha \und \alpha \sub \tau^-_\alpha$ in the list. Invoking transitivity and normalising can only yield stronger bounds, i.e.\@ the process is monotone under the order $\preceq$ defined as:\footnote{Introducing another subsumption-like relation might be surprising, but $\preceq$ is more useful here as it is syntactic (checks bounds) rather than semantic (as $\subsume$, which checks instantiations). It is also a stronger condition: $\preceq$ implies $\subsume$.}
$$ \hat c' \preceq \hat c \iff \hat c' = \fals \text{ or } (\forall \alpha \ldotp \pi_\alpha^+ \sub \tau_\alpha^+ \text{ and } \tau_\alpha^- \sub \pi_\alpha^-) $$
where $\tau_\alpha$ and $\pi_\alpha$ are the bounds on a variable $\alpha$ in $\hat c'$ and $\hat c$, respectively.
While it is expected this process terminates \cite{pottier-framework, simple-sub, mlstruct} -- determining the constraint is satisfiable -- I conjecture it and only verify it in practice:
\begin{conjecture}
    For any normalised constraint $\hat c$, applying transitivity at each $\alpha$ and leads to some $\hat c'$ such that $\hat c' \preceq \hat c$. This process eventually reaches a fixed-point. We call this fixed-point the \emph{solved constraint}, and denote it by $\mathcal S(c)$ for any initial $c$.
\end{conjecture}

\begin{example}
    Denoting applications of transitivity by $\Rightarrow$, consider two examples:
    \begin{align*}
        & \flext{\mathrm{quack}: \beta}{\flabsent} \sub \alpha \und \alpha \sub \flext{\mathrm{quack}: \top \to \top }{\flftop} \\
        \Rightarrow\;& \flext{\mathrm{quack}: \beta}{\flabsent} \cdecomp \flext{\mathrm{quack}: \top \to \top }{\flftop} = \gamma \sub \top \to \top
    \end{align*}
    where we obtained a bound on the type $\beta$ of the field $\mathrm{quack}$.
    $$ \top \to \top \sub \gamma \und \gamma \sub \flrec{} \;\implies\; \top \to \top \cdecomp \flrec{} = \fals $$
    where we reached a contradiction on satisfying bounds on $\gamma$.
\end{example}

\subsection{Solutions}
\label{subsec:simplification}

We now consider the question of extracting the type scheme from a solved constraint. Given a generated constraint $\denot{e : \tau}$, if it is solved successfully we have:
$$ \mathcal S \left( \denot{e : \tau} \right) = \overline{\exists \alpha \ldotp {}} \sbnd $$
in which case we (abusing notation with $W$) return the type scheme
$$ \sigma =  \forall W \ldotp \tau $$
It is important to simplify type schemes -- however, I do not propose novel ways to do this, and entirely refer to \textcite{simple-sub} and \textcite{mlstruct}. In practice, I have found that to ensure efficient termination I had to simplify the CNF (or DNF) forms. Inlining bounds (particularly done by \textcite{dolan-thesis}) and removing redundant type variables mainly serves to improve readability.

\begin{example}
    Given:
    $$ e =  $$
    we have:
    $$ \mathcal S(\denot{e : \tau}) = $$
    thus, $\Gamma \vdash e : \sigma$ at:
    $$ \sigma = \forall \ldotp  $$
    which can be equivalently (i.e.\@ $\sigma \subsume \sigma' \subsume \sigma$) simplified to $\sigma'$:
    $$ \sigma' = \forall \ldotp $$
\end{example}\todo[color=blue]{ex}

\section{Breaking records: Homomorphism extension}
\label{sec:morphisms}

The crucial issue identified early on by \textcite{operations-on-records} with typing extensible records using subtyping is the \emph{update problem}. We exemplify it on an FL program using record extension:
$$ e = \fllam x \fllam y \flext{\ell = y}{x} $$
which would naively be given the (non-principal) type scheme
$$ \cdot \vdash e : \forall \alpha \ldotp \flext{\ell : \flabsent}{\flftop} \to \alpha \to \flext{\ell : \alpha}{\flftop} $$
causing \emph{loss of information} about the non-$\ell$ fields of $x$. There is no obvious way in which just subtype and parametric polymorphism can capture that information -- hence, the standard solution to the problem is row polymorphism \cite{remy-records}, as row type variables can stand for other fields. 

We might prefer to avoid adding rows to the system -- avoiding the inherent complexity -- and to stick to just bounded parametric polymorphism. One approach is to update a specific field in the record type underlying a type variable \cite{operations-on-records} via a \enquote{type-function} $\mathrm{update}_\ell(\tau, \phi) = \tau$, like:
$$ \qquad \cdot \vdash e : \forall \alpha, \rho \sub \flext{\ell : \flabsent}{\flftop} \ldotp \rho \to \alpha \to \mathrm{update}_\ell(\rho, \alpha) \qquad \boxed{?!} $$
Indeed, it would be convenient to have \emph{metafunctions on types} to specify such type schemes.

I propose to follow the algebraic approach and exploit homomorphisms in the type algebra -- such well-behaved (meta)functions -- for this purpose. Thanks to homomorphism laws, we are able to successfully solve constraints involving homomorphism applications. Hence, they uncover a new direction in the design of type systems applying algebraic subtyping.

\begin{example}
    For typing extensible records, we shall develop a homomorphism $\mathrm{forget}_\ell$, which given a record type sends a field $\ell$ to $\flftop$.
    By sending the field $\ell$ to $\flftop$, we can intersect the result with a singleton record of $\ell$ to set it to a wanted field type ($\forall \phi \ldotp \flftop \meet \phi = \phi$). 
    For example:
    \setlength{\tabcolsep}{0pt}
    $$\begin{array}{rll}
        &\hspace{-0.9em}\tau &\;\hspace{-0.9em}= \flext{\mathrm{foo}: \flabsent}{\flabsent} \\
        \mathrm{forget}_\mathrm{foo}( &\hspace{-0.9em}\tau ) &\;\hspace{-0.9em}= \flext{\mathrm{foo}: \flftop}{\flabsent} \\
        \mathrm{forget}_\mathrm{foo}( &\hspace{-0.9em}\tau ) \meet \flext{\mathrm{foo}: \alpha}{\flftop} &\;\hspace{-0.9em}= \flext{\mathrm{foo}: \alpha}{\flabsent}
    \end{array}$$
    This shows how we will replace an absent field in a record with one present with $\alpha$.

    $\mathrm{forget}_\ell$ is reminiscent of the retraction operator on record types of \textcite{operations-on-records}.
\end{example}

\begin{figure}
    \centering
    $$\boxed{\tau \typeq \tau}$$
    \vspace{-2em}
    \begin{align*}
        \morph(\tau \meet \pi) &\typeq \morph(\tau) \meet \morph(\pi) \\
        \morph(\tau \join \pi) &\typeq \morph(\tau) \join \morph(\pi) \\
        \morph(\top) &\typeq \top \\
        \morph(\bot) &\typeq \bot
    \end{align*}
    \vspace{-1em}
    $$ \equivctx ::= \cdots \mid \morph(\equivctx) $$
    \caption{Homomorphism laws in the Boolean algebra of types, which extend the definition of equivalence $\tau \typeq \tau$. Note that $\morph(\comp \tau) \typeq \comp \morph(\tau)$ follows from these laws (by a routine check of complement axioms).}
    \label{fig:morphism-laws}
\end{figure}

\subsection{Signature}

We begin developing the idea formally by extending the syntax of types with \textbf{homomorphism applications} for some homomorphisms $\morph$ specific to a language, and including a homomorphism $\morph = \mathrm{id}$:
$$ \tau ::= \cdots \mid \morph(\tau) $$
requiring that equivalence of types respects the homomorphism laws (Figure \ref{fig:morphism-laws}) and that $\mathrm{id}(\tau) \typeq \tau$. 

In order to solve constraints in the presence of homomorphism applications, we require they have an \emph{adjoint-like}\footnote{Or, perhaps better said, Galois connection-like -- though adjoint is a convenient name.} structure given by left- and right-adjoint homomorphisms $\ladj \morph$ and $\radj \morph$ for all $\morph$. In practice, not all homomorphisms have an adjoint structure (which is why we call this data adjoint-like), thus we also allow \emph{remainder} constraint functions $\ladjrest \morph(\tau) = c$ and $\radjrest \morph(\tau) = c$, so that we have constraint equivalences:
\begin{align*}
(\morph(\tau) \sub \pi) \cstreq \left(\tau \sub \radj \morph(\pi) \und \radjrest \morph(\pi) \right) \\
(\tau \sub \morph(\pi)) \cstreq \left(\ladj \morph(\tau) \sub \pi \und \ladjrest \morph(\tau) \right)
\end{align*}
The use of $\ladjrest \morph$ and $\radjrest \morph$ makes it clear why we call these remainders: in the degenerate case that they introduce no further constraint ($\ladjrest \morph = \radjrest \morph = \tru$), $\ladjrest \morph$ and $\radjrest \morph$ are precisely left and right Galois connections to $\morph$.

\begin{example}
    In order to give adjoint-like structure $\mathrm{forget}_\ell$, we need a \enquote{dual} for it. We call this morphism $\mathrm{free}_\ell$, and define it so that it sends a field $\ell$ to $\flfbot$. Then we have the following adjoints and remainders:
    $$\begin{array}{c|cccc}
    \morph & \ladj \morph & \ladjrest \morph(\tau) & \radj \morph & \radjrest \morph(\tau) \\ \hline 
    \mathrm{forget}_\ell & \mathrm{free}_\ell & \tru & \mathrm{id} & \flext{\ell : \flftop}{\flfbot} \sub \tau \\
    \mathrm{free}_\ell & \mathrm{id} & \tau \sub \flext{\ell : \flfbot}{\flftop} & \mathrm{forget}_\ell & \tru \\ 
    \mathrm{id} & \mathrm{id} & \tru & \mathrm{id} & \tru 
    \end{array}$$
    $\mathrm{free}_\ell$ and $\mathrm{forget}_\ell$ form an \enquote{adjoint pair} (inspiring the naming, cf.\@ free-forgetful adjunction).
\end{example}

\subsection{Constraint solving}

We now extend the constraint solving process of Section \ref{sec:constraints} with support for homomorphism applications. This is relatively straightforward: we only need to amend the construction of variable bounds from subtyping constraints. Thus, we firstly amend the syntax of $\typvar$ (in $\typcnf$):
$$ \typvar ::= \morph(\alpha) \mid \comp  \morph(\alpha) $$
which generalises $\typvar$ (at $\morph = \mathrm{id}$).
Crucially, we can still construct the CNF thanks to the fact that $\morph$ are homomorphisms: we can we just \emph{push down} all applications.

For constructing variable bounds, only the final cases are affected -- we need to show $\top \sub \typvar \join \tau$ gives a variable bound. We sketch how to do this by exploiting the adjoint-like structure (via appropriate $\step$ rules):
\begin{itemize}
    \item If $\typvar = \morph(\alpha)$, then $\top \sub \morph(\alpha) \join \tau \stackrel{\text{swap}}{\step} \comp \tau \sub \morph(\alpha) \stackrel{\text{left-adj}}{\step} \ladj \morph(\tau) \sub \alpha \und \ladjrest \morph(\tau) $.
    \item If $\typvar = \morph(\comp \alpha)$, then $\top \sub \comp \morph(\alpha) \join \tau \stackrel{\text{swap}}{\step} \morph(\alpha) \sub \tau \stackrel{\text{right-adj}}{\step} \alpha \sub \radj \morph(\tau) \und \radjrest \morph(\tau)$.
\end{itemize}
where we proceed recursively on any remainder constraints. 
Note that constraints $\top \sub \typvar \join \tau$ remain always satisfiable, since we can always set $\typvar$ to $\top$ (since $\morph(\bot) \equiv \bot$, $\morph(\top) \equiv \top$).

\subsection{Typing extensible records}
Now that we have describe the use of type homomorphisms in constraint solving, we present the promised concrete application -- typing extensible records, presented for Featherweight Lua. To this end, we define the homomorphisms $\mathrm{forget}_\ell$ and $\mathrm{free}_\ell$ (introduced in examples):
$$ 
\dfrac
  {\tau = \flext{\ell' : \phi_{\ell'},  \overline{\ell : \phi_\ell}}{\phi}}
  {\mathrm{forget}_\ell(\tau) = \flext{\ell' : \flftop,  \overline{\ell : \phi_\ell}}{\phi}}
\qquad 
\dfrac
  {\tau = \flext{\ell' : \phi_{\ell'},  \overline{\ell : \phi_\ell}}{\phi}}
  {\mathrm{free}_\ell(\tau) = \flext{\ell' : \flfbot,  \overline{\ell : \phi_\ell}}{\phi}}
$$
The key property we make use of is that we can give a rule equivalent to the \emph{rule scheme} \textsc{FL-Typ-Ext} -- for which there is no clear way to generate constraints (while avoiding the update problem \cite{operations-on-records}). We instead use:
$$ 
\irule{FL-Typ-HExt}{\Gamma \vdash e : \tau \meet \flext{\ell : \flabsent}{\flftop} \quad \Gamma \vdash e' : \tau'}{\Gamma \vdash \flext{\ell = e'}{e} : \mathrm{forget}_\ell(\tau) \meet \flext{\ell : \tau'}{\flftop}}
$$
\textsc{FL-Typ-HExt} makes it straightforward to generate the necessary constraints (see Appendix \ref{extra:fl-constraints} for details).
% (sketch of the proofs of homomorphisms laws and equivalence of \textsc{FL-Typ-Ext} and \textsc{FL-Typ-HExt})
\todo[color=green]{extra}
Thus, I have solved the record update problem under algebraic subtyping without row polymorphism.

\section{Correctness}
\label{sec:correctness}

I state correctness theorems for \inference{}, particularly for \emph{solved constraints} $\mathcal S(c)$. We begin with soundness:
\begin{theorem}[Soundness]
    For all $\psi$ and $c$, $\psi \vdash \mathcal S(c)$ if and only if $\psi \vdash c$.
\end{theorem}
\begin{proof}
    Follows directly by semantic preservation of $c$, since $c \leadsto^* \mathcal S(c)$ by construction.
\end{proof}

However, I do not provide proofs for the following -- and only provide evidence based on the prototype implementation (Chapter \ref{fabric}), arguing that \inference{} generalises prior work where these hold. 

\begin{conjecture}[Termination]
    The constraint solving $\mathcal S(c)$ in Section \ref{sec:constraints} always terminates.
\end{conjecture}

\begin{conjecture}[Completeness]    
    For all $c$, $\mathcal S(c) \ne \fals$ if and only if $\exists \psi \ldotp \psi \vdash c$.
\end{conjecture}

\begin{conjecture}[Principality]
    Returned type schemes are minimal under both $\subsume$ and $\preceq$.
\end{conjecture}

Note that: \begin{itemize}
    \item Termination requires appropriate choices of $\cdecomp$, $\ladjrest \morph$/$\radjrest \morph$, and ensuring the fixed-point is found after a finite number of iterations.
    \item Completeness relies on the transitive closure finding all possible contradictions in the system -- which is a standard result in the work of \textcite{pottier-framework} and in constraint solving folklore.
    \item Principality for $\preceq$ seems straightforward (by construction of the closure), and leads to principality for $\subsume$ as $\preceq$ is a stronger condition.
\end{itemize}
We do not consider decidability of subsumption of type schemes (to the well-founded dismay of \textcite{dolan-thesis}). I rely on the claim of \textcite{mlstruct} that it is resolvable by solving an appropriate constraint.

\section{Conclusions}
\label{sec:conclusions}

I have described \inference{} -- a language-agnostic, constraint-based type inference framework based on algebraic subtyping, as introduced by \textcite{mlsub}.
It soundly infers bounded parametric type schemes in the presence of structural subtyping. 
I use elegant Boolean algebraic techniques -- as set out by \textcite{mlstruct} -- which I extend with the use of \emph{homomorphisms}. 
My description lends itself directly to an implementation -- this implementation is part of my deliverable, described in the following chapter.



  \addtocontents{toc}{\newpage}
\chapter{Design and Implementation of Fabric}
\label{fabric}

In this short chapter, I present \textbf{\fabric{}} -- a functional programming language with structural subtyping. 
\fabric{}'s role in my thesis is roughly similar to that of \mlsub{} \cite{dolan-thesis}: my design (Section \ref{sec:fabric-design}) is driven by: \begin{itemize}
    \item Desire to statically type features present in dynamically typed languages, like Python or Lua.
    \item Applying \inference{} (Chapter \ref{algebraic-subtyping}) -- making \fabric{} an example of its expressive power.
\end{itemize}
Furthermore, in Section \ref{sec:fabric-impl} I explore the implementation of my \fabric{} compiler -- \compiler{} -- targeting \wasm{}, which includes a complete implementation of \inference{}. I thus report my practical experience working with \wasm{} and \inference{}.

\section{Design}
\label{sec:fabric-design}

The basis of \fabric{} is Chapter \ref{static-soul}'s Featherweight Lua, extended with more record operations, variants, tuples, and \emph{nominal type abbreviations}. \fabric{} is essentially a superset of \mlsub{} \cite{mlsub} -- I consider Dolan's proposed extensions \cite[Chapter~9]{dolan-thesis} -- and is influenced by \mlstruct{} \cite{mlstruct}. 
A formal development for \fabric{} is given in Appendix \ref{extra:fabric}.

% Due to time (and space) constraints, I only state type safety (Progress and Preservation) for \starr{} (Chapter~\ref{star}; Section \ref{subsec:type-safety}). However, \fabric{}'s design is also intended to admit type safety.

Selected \fabric{} programs written in code accepted by \compiler{} are given in Figures~\ref{fig:fabric-example-eval}~and~\ref{fig:fabric-example-pairwise}.


\begin{figure}[p]
    \centering
    \begin{cminted}{ocaml}
(* strict Y fixed-point combinator *) 
let fix = f => let z = x => f (v => x x v) in z z in
let eval = fix (eval => e =>
  match e with 
  | Add r => (eval r.fst) + (eval r.snd)
  | Mul r => (eval r.fst) * (eval r.snd)
  | _ r => r.default (* match on any tag *)
) in eval (
  Add { 
    fst: Mul { 
      fst: Lit { default: 2 },
      snd: Lit { default: 3 } 
    }, 
    (* different tag and redundant field *) 
    snd: Var { default: 1, foo: {} }
  }
)
\end{cminted}
    \caption{\fabric{} program implementing a recursive function for evaluating arithmetic expressions, where some nodes return a \enquote{default} value. Code generated by \compiler{} correctly computes \mintinline{ocaml}{7}. \inference{} infers a polymorphic type scheme for \mintinline{ocaml}{eval}, \eg{} rejecting \mintinline{ocaml}{eval (Lit {})}. Note $\fllam x e$ is written \texttt{x => e}.}
    \label{fig:fabric-example-eval}
\end{figure}

\begin{figure}[p]
    \centering
    \begin{cminted}{ocaml}
(* infers that the result is of even length *)
let stutter = xs => 
  match xs with
  | Nil _ => Nil ()
  | Cons c => Cons {
      head: c.head,
      tail: Cons { 
        head: c.head, tail: stutter c.tail 
      }
    } 
(* infers that xs must be of even length *)
in let pairwise = xs => 
  match xs with
  | Nil _ => Nil () 
  | Cons a => (
      match a.tail with 
      | Cons b => Cons { 
          head: (a.head, b.head), tail: pairwise b.tail 
        })
in (xs => pairwise (stutter xs), (* fine! *)
    xs => pairwise (Cons { head: 0, tail: stutter xs })) (* type error! *) 
\end{cminted}
    \caption{\fabric{} program implementing functions \texttt{stutter} (duplicating elements in a list, doubling its length) and \texttt{pairwise} (pairing adjacent elements of an \textbf{even-length} list). \inference{} detects an error: \texttt{pairwise} is passed an odd-length list. Inferring such properties with algebraic subtyping is explored by \citeauthor*{structural-refinement-types}~\cite{structural-refinement-types}.}
    \label{fig:fabric-example-pairwise}
\end{figure}

\subsubsection{Data types}

I focus on describing records and variants, though \fabric{} also has tuples and integers.
\begin{description}
\item[Records]
Beyond FL, I add field restriction $e \fbrestricted \ell$ \cite{operations-on-records} and checked-projection $\fbcheck{e}{\ell}$ (accompanied by an \enquote{optional} field type, $?\boxed{\tau}$), inspired by operations for objects in dynamic languages.

In contrast to \textcite{mlstruct}, I do not use \textsc{Forsythe}-style singleton record types \cite{forsythe}, as they lead to a confusing type lattice.\footnote{\eg{}: in \mlstruct{} $\flrec{x : \alpha} \join \flrec{y: \beta} = \top \implies \comp \flrec{ x : \alpha } \sub \flrec{ y : \beta } $, as \emph{there is no empty record type} \cite[Section~4.4.5]{mlstruct}.} Instead, I rely on the same \emph{partial-function} strategy for representing record (and variant) types as in FL (Section \ref{subsec:featherweight-lua}).

\item[Variants] Following suggestions of \textcite[Section~9.2]{dolan-thesis}, I add variants $\fbtag{T}{e}$ to \fabric{} (and a $\mathrm{match}$), though not in the form of his \emph{tagged records}. To address his motivating example, I instead propose \emph{untagging} -- pattern matching on \emph{any} tag -- easily type checked with my syntax of variant types.
\end{description}

\subsubsection{Patterns}
Structural typing invites a rich pattern-matching facility \cite{parreaux-patterns}. I provide a sketch of a design for \fabric{} in Appendix \ref{extra:fabric}, using conservative typing rules to ensure exhaustive matches. 

\subsubsection{Nominal types} 
Combining both nominal and structural types is desirable \cite{mlstruct, integrating-nominal-and-structural}. In \fabric{}, I propose a method inspired by OCaml's \href{https://ocaml.org/manual/5.1/privatetypes.html#ss:private-types-abbrev}{private type abbreviations}.
\emph{Nominal abbreviations} $\nom$ in \fabric{} are declared via $\mathrm{type}\,\nom=\tau\,\mathrm{in}\,e$. They are introduced and eliminated via \textbf{abbreviation-casts} $\fbcast{e}{\tau}{\tau'}$, well-typed exactly when $\tau$ is equal to $\tau'$ \emph{up to abbreviations}, and $e$ is of type $\tau$. The intermediate type $\tau$ is key for type checking.

The same problem is addressed by \textcite{mlstruct}. Their approach -- \emph{nominal tags} (reminiscent of Dolan's \emph{tagged records}) -- \emph{refines} existing structural types, and requires dedicated runtime support.

This strategy of adding nominal types to the system could lead to integrating algebraic subtyping with higher-kinded types and higher-rank polymorphism (as set out by \textcite[Section~11.1]{dolan-thesis}). 

\section{Implementation}
\label{sec:fabric-impl}

I used OCaml for implementing \compiler{}. My main dependencies are: \href{https://opensource.janestreet.com/core/}{Core}, \href{https://github.com/janestreet/ppx_jane}{\texttt{ppx\_jane}} (pre-processor macros, S-expressions), \href{https://github.com/mirage/alcotest}{\texttt{alcotest}} (test framework), and \href{https://github.com/inhabitedtype/angstrom}{\texttt{angstrom}} (parser combinators).

The implementation is complete with minor omissions.\footnote{Mainly: compound pattern matching, nominal types, and code generation for recursive let-bindings (but not the Y combinator).} For brevity, I describe interesting aspects of type inference (Section \ref{subsec:type-inference-impl}) and code generation (Section~\ref{subsec:codegen}).

\subsection{Type inference} 
\label{subsec:type-inference-impl}

I devised an implementation of \inference{} usable with any type system implementing its signature -- not just for \fabric{}, but for any language for which we can generate constraints.

\subsubsection{Exploiting the module language} The framework was particularly satisfying to implement using OCaml's module language. Firstly, the framework's signature is given by an module \emph{signature} (given in Figure \ref{fig:inference-signature}). Type systems are specified by an \emph{implementation} module satisfying the signature, from which a \emph{functor} derives the constraint solver.

\begin{figure}[p]
    \centering
    % \begin{tabular}{c}
% \begin{ocaml}
\begin{cminted}{ocaml}
module type TypeSystem = sig
  type 'a typ  (* type constructors with holes 'a *)
  type arrow   (* homomorphisms *)

  (* mapping over type constructor subterms *)
  val map : f:('a -> 'b) -> 'a typ -> 'b typ
  (* type constructor lattice *)
  val top : 'a typ
  val bot : 'a typ
  val join : 'a lattice -> 'a typ -> 'a typ -> 'a typ
  val meet : 'a lattice -> 'a typ -> 'a typ -> 'a typ
  (* type constructor decomposition *)
  val decompose : ('a typ * 'a typ) -> ('a * 'a) list Or_error.t

  module Arrow : sig
    type t = arrow

    (* identity homomorphism *)
    val id : t
    val is_id : t -> bool
    (* composition *)
    val compose : t -> t -> t
    (* application to type constructors *)
    val apply : (t -> 'a -> 'a) -> t -> 'a typ -> 'a typ

    (* adjoint structure *)
    val swap_left : ('a typ -> 'a) -> t -> t * 'a
    val swap_right : ('a typ -> 'a) -> t -> t * 'a
  end
end
\end{cminted}
% \end{ocaml}
% \end{tabular}
    \caption{The key parts of the OCaml module signature of a type system suitable for type inference.}
    \label{fig:inference-signature}
\end{figure}

\subsubsection{Type language representations} \inference{}'s OCaml signature is polymorphic with respect to different type languages. 
% This is because we have to change between different representations of types. 
Usually, we use the full type language (following \inference{}'s $\tau$), represented using the OCaml type \texttt{Alg.t}. On the other hand, in constraint solving we rely on the use of normal forms -- the clause-based CNF/DNF (\texttt{CNF.t}/\texttt{DNF.t}). As CNF--DNF conversion can blow up the number of clauses, we manipulate constraints as $\tau_\mathrm{dnf} \sub \tau_\mathrm{cnf}$. To avoid code duplication, I implement DNF in terms of CNF, exploiting the duality between them. 

% \subsubsection{Constraint solving} As advertised, the implementation of constraint solving is close to the technical description in Section \ref{sec:constraints}. It is infeasible to solve constraints directly by pattern matching on $\tau$/\texttt{Alg.t} due to mismatched-polarity cases like $\alpha \join \beta \sub \gamma \meet \delta$.

\subsubsection{Simplification of type schemes} In Section \ref{subsec:simplification} I hinted at possible approaches to simplifying type schemes. I implemented most heuristics proposed by \textcite{simple-sub} and \textcite{mlstruct}.\footnote{Notably, I omitted \emph{unification of indistinguishable variables} and \emph{hash-consing}.} 
% \begin{itemize}
%     \item Simplification of CNF/DNF forms: removing subsumed clauses and clauses equal to $\top$/$\bot$.
%     \item Inlining bounds: replacing a type variable with its bound if it only occurs in co/contravariant positions.
%     \item Sandwich inequalities: if we find $\tau \sub \alpha \sub \tau$, then we can substitute $\alpha$ for $\tau$.
% \end{itemize}
However, this subset has proven unsatisfactory, often returning redundant type variables that could be simplified by-hand. This limitation is mainly due to the time-consuming implementation. 
Besides Parreaux's other heuristics, it would be interesting to try automata-simplification of \textcite[Chapter~7]{dolan-thesis}.

\subsubsection{Constraint generation} I devised a simple DSL for constraint generation (example in Figure \ref{fig:cstr-gen}), using OCaml's let-binding operators for composing constraints and introducing fresh type variables, giving the code a declarative feel. 

\begin{figure}
    \centering
\begin{cminted}{ocaml}
Let (p, e, e') ->
  let* xs, t = pat p in
  let* e = go env e in
  let* () = e <: t in
  go (push xs env) e'
\end{cminted}
    \caption{Example of constraint generation for \fabric{} let-bindings ($\fllet p e e'$). Here, \texttt{go env e} stands for a type $\tau$ such that $\texttt{env} \vdash \texttt{e} : \tau$ (under some carried constraint) -- a slight deviation from \inference{}'s specification.}
    \label{fig:cstr-gen}
\end{figure}

\subsection{Code generation} 
\label{subsec:codegen}

In order to execute \fabric{} code, I sought to generate code that could be lowered directly to machine code. I chose to go with \wasm{} \cite{wasm} and to use the \binaryen{} \cite{binaryen} toolchain, because it is modern, stable, relatively high-level, and well-documented.\footnote{I also considered other targets -- C, Lua, .NET or JVM bytecode, or LLVM. Ultimately, I thought using \wasm{} would be easiest and most interesting.} Furthermore, I wanted to explore its new extension with garbage collection \cite{wasm-gc}, enabling automatic memory management for \fabric{}.

\binaryen{} is only accessible from OCaml using the C~API.\footnote{There are ready solutions -- like \href{https://github.com/grain-lang/binaryen.ml}{\texttt{binaryen.ml}} -- but outdated (\eg{} no \textsc{WasmGC}) or broken with newer \binaryen{} versions.} I used the brilliant work of \textcite{ocaml-ctypes} on \href{https://github.com/yallop/ocaml-ctypes}{\texttt{ocaml-ctypes}} to produce my own, type-safe bindings.

Invoking bindings directly led to clumsy, imperative code. Inspired by \textcite{offshoring-c}, I created a DSL on top of \binaryen{} dubbed \binaryendsl{}. Its most prevalent abstraction is \texttt{Cell} (Figure \ref{fig:cell-def}), which encapsulates the different types of storage available in \wasm{}, making it easier to write generic helpers for code generation.
I give a basic example of using \binaryendsl{} in Figure \ref{fig:binaryer-example}. 

I used \binaryendsl{} to implement code generation for almost all of \fabric{}. I included a suite of unit tests using a basic printing function, \texttt{\%print\_i32}, provided by \binaryen{}.

\begin{figure}[p]
    \centering
    \begin{cminted}{ocaml}
type loc = Cell0.loc =
| Local of { idx : T.Index.t }
| Global of { name : string; mut : bool; handle : T.Global.t }
| Table of { name : string; idx : T.Expression.t }
| Address of {
    addr : T.Expression.t;
    size : uint32;
    offset : uint32;
    align : uint32;
    mem : string;
  }
| Struct of {
    target : T.Expression.t;
    struct_type : T.HeapType.t;
    field_idx : T.Index.t;
  }
| Array of {
    target : T.Expression.t;
    array_type : T.HeapType.t;
    idx : T.Expression.t;
  }

type t = { typ : T.Type.t; loc : Cell0.loc }


(* val ( ! ) : t -> T.Expression.t *)
(* val ( := ) : t -> T.Expression.t -> T.Expression.t *) 

\end{cminted}
    \caption{Definition of \texttt{Cell.t} -- abstracting over locals, globals, tables, memory addresses, structure fields, and array elements. I also give the signature of its two primitives: read (\texttt{!}) and write (\texttt{:=}). 
    % \texttt{T.<thing>.t} is the OCaml type for \binaryen{}'s representation of \texttt{<thing>}s.
    }
    \label{fig:cell-def}
\end{figure}

\begin{figure}[p]
    \centering
    \begin{cminted}{ocaml}
(* Set up Binaryen context *) 
let (module Ctx) = context () in
let open Ctx in
feature C.Features.reference_types;
feature C.Features.gc;
Memory.set ~initial:10 ~maximum:10 ();
(* Declare this function may print *)
Function.import "print_i32" "spectest" "print_i32" Type.int32 Type.none;
(* Define the main function *)
let main =
  Function.make ~params:Type.none ~result:Type.none (fun _ ->
    let open Cell in
    (* Type foobar_t of a structure with fields foo and bar *)
    let foobar_t =
      Struct.t Type.[ ("foo", field ~mut:true int32); ("bar", field int32) ]
    in
    (* Define a local reference and access it as a foobar_t *)
    let q = local Type.anyref in
    let q_foo = Struct.cell foobar_t !q "foo" in
    let q_bar = Struct.cell foobar_t !q "bar" in
    Control.block
      [
        q :=
          Struct.make foobar_t
            [ ("foo", Const.i32' 42); ("bar", Const.i32' (1337 - 42)) ];
        q_foo := Operator.I32.(!q_foo + !q_bar);
        Function.call "print_i32" [ !q_foo ] Type.none;
      ])
in
(* Mark the main function *)
Function.export "main" main;
Function.start main;
assert (validate ());
interpret ();    
\end{cminted}
    \caption{Basic program defined using \binaryendsl{}. Under the hood, it calls \binaryen{}'s API, producing \wasm{} code (given in Appendix \ref{extra:codegen}) which prints \texttt{1337} at runtime.}
    \label{fig:binaryer-example}
\end{figure}

\subsubsection{Runtime representation under subtyping}

It is well-known that structural typing makes the problem of efficiently representing values at runtime more difficult, as a value might be used as any supertype \cite{tapl}. 

This problem also arises for \fabric{}'s record and variant types: 
\begin{description}
    \item[Variants] I implemented the same approach as \textcite{polymorphic-variants} -- variant values are represented as a pair of a 32-bit hash of its tag and a reference to its payload.
    \item[Records] Due to time constraints, I did not experiment with record representations, and stuck to a na\"ive one (with linear-time projection). One interesting direction would be the work of \textcite{remy-extensible-records}, where a record is a hash table with a pre-computed hashing function (so projection is constant-time). 
\end{description}

\needspace{7em}
\section{Conclusions}

I summarise my experience with the three main areas covered in this chapter: the use and implementation of \inference{}, targeting \wasm{}, and the design of languages for algebraic subtyping. 
% I also give some more code examples in Appendix \ref{extra:fabric}.\todo[color=green]{extra}

\paragraph{Experience with \inference{}}
% His work resulted in \textsc{Inferno}, a framework for constraint-based Hindley-Milner type inference -- I performed a similar task in a setting with subtyping. 
By separating concerns of the type and constraint languages in my implementation of \compiler{}, I was able to experiment with type inference for not only \fabric{}, but also \starr{} (Chapter \ref{star}) -- confirming the benefits outlined by \textcite{pottier-framework}. I have also seen the importance of complete type scheme simplification. 

\paragraph{Experience with \wasm{}}
While the \wasm{} tooling for high-level languages is limited, good bespoke solutions are possible. However, \binaryen{} had poor error diagnostics for invalid programs, making debugging difficult. \wasm{}'s formal specification and stability are great boons towards its practical use.
I observed that the current design of automatically managed reference types is limiting in terms of achievable performance, \eg{} making some efficient memory representations impossible~\cite{double-ended-bit-stealing}.

\paragraph{Language design with algebraic subtyping} Constructing a lattice of types was helpful in preventing misbehaved type system features, and did not prove to be a constraint. Rapidly experimenting with new features using \inference{} was liberating during design.

  \chapter{Structuring Arrays with Algebraic Shapes}
\label{star}

\section{Background}

\section{Motivation}

\section{Calculus}

\section{Typing}

\section{Examples}

\section{Conclusions}

  \chapter{Conclusions}
\label{conclusions}

I set out to show that structural subtyping is an interesting and rich approach to designing flexible type systems. I addressed this with techniques from both theory of programming languages and practical design.

From the theoretical side, I demonstrated that structural subtyping takes us closer towards the flexibility of dynamic languages by giving the FJ-FL translation (Chapter~\ref{static-soul}). 
I built a constraint-based type inference framework for languages with structural subtyping, \textbf{\inference{}}, deriving from \emph{algebraic subtyping} (Chapter~\ref{algebraic-subtyping}). 
It generalises prior work and has a simpler presentation, giving clear requirements for the type system.
By extending the type language of existing systems in a novel way (with \emph{homomorphism applications}), I unlocked a new solution to the record update problem. 
Thus, I widened the range of language features that can be conveniently type-checked without annotations.

From the practical side, I designed a functional language with structural subtyping -- \textbf{\fabric} (Chapter~\ref{fabric}). 
I implemented a compiler targeting \wasm{} for it -- \textbf{\compiler{}} -- including my type inference framework (\inference{}).
Lastly, I gave a novel array calculus design -- \textbf{\starr{}} (Chapter~\ref{star}). It uniquely features array shapes and indices that exploit structural subtyping, providing safety while admitting type inference with \inference{}. It thus serves as a new middle-ground between the conventional untyped discipline (flexible -- but unsafe), and complicated (but safe) dependently-typed systems. 

\paragraph{Publications} I gave a talk about my thesis at BCTCS 2025. A research paper on \starr{} co-authored with my supervisors was accepted for publication in the ARRAY Workshop Proceedings.

\paragraph{Further work}
I identify the following directions for further work as most interesting: \begin{itemize}
    \item Fully proving the correctness of constraint-based algebraic subtyping presentation -- particularly in presence of homomorphism applications. 
    \item Determining the scope of type system features expressible using homomorphisms, besides extensible records (FL) and type isomorphisms (\starr{}).
    \item With \inference{} developed, it would be exciting to use it to for typing real-world dynamic languages, like Python or Lua (as done with set-theoretic types \cite{set-theoretic-types-for-elixir, set-theoretic-types-for-erlang, castagna-dynamic}).
    \item Implementing \starr{} in practice (\eg{} extending Futhark \cite{futhark}, embedding in Python \cite{ein}), which leads to interesting problems: performant code generation, shape inference, and recursive indices.
\end{itemize}

Despite the age of this natural idea, I have shown structural subtyping should be revisited as an effective way for designing practical type systems. Likewise, it still leads to many interesting theoretical questions.
  
%TC:ignore 

  % This ensures that the subsequent sections are being included as root
  % items in the bookmark structure of your PDF reader.
  \bookmarksetup{startatroot}

\begingroup
  \backmatter
  \printindex
  \printbibliography
\endgroup

\addtocontents{toc}{\protect\setcounter{tocdepth}{1}}
\setcounter{chapter}{1}
\appendix 

\begingroup
  \let\clearpage\relax
  \renewcommand{\nomname}{Notation}
  \printnomenclature
\endgroup

  \chapter{Proof of Type Safety for Star}
\label{star-type-safety-proof}

%TC:endignore 
\end{document}